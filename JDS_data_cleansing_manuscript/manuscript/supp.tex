\documentclass[linenumber]{jdsart}
\pdfminorversion=7

%% \usepackage{amsmath, amssymb, graphicx, booktabs, url, hyperref}
%% \usepackage{setspace}
%% \onehalfspacing
%% \usepackage[margin=1in]{geometry}

\usepackage{siunitx}
\sisetup{
    group-separator = {,},
    round-mode = places,
    round-precision = 2,
    output-decimal-marker = {.},
    table-number-alignment = center,
    table-figures-integer = 6,
    table-figures-decimal = 2,
    table-figures-uncertainty = 2
}

\usepackage{comment}
\usepackage{multicol}
%\usepackage{fontawesome5}
%\usepackage{booktabs, textgreek}
\usepackage{booktabs}
\usepackage{rotating}
\usepackage{subcaption}
\usepackage{enumitem}
\usepackage{dirtree}
\usepackage{tikz}
\usetikzlibrary{positioning}

% DO NOT load hyperref here — jdsart already does it
% \usepackage{hyperref}

% image path
\graphicspath{{./}{./supplemental_images/}}

% =========================================================
% AUTO SUPPLEMENT NUMBERING (Tables/Figures as S1, S2, ...)
% =========================================================
\makeatletter
\renewcommand{\thefigure}{S\arabic{figure}}
\renewcommand{\thetable}{S\arabic{table}}
\renewcommand{\fnum@table}{\tablename~\thetable}
\renewcommand{\fnum@figure}{\figurename~\thefigure}
\makeatother

%% float control
\renewcommand\floatpagefraction{0.75}
\setcounter{totalnumber}{50}
\setcounter{topnumber}{50}
\setcounter{bottomnumber}{50}

% hyperref settings (safe to keep)
\hypersetup{
  colorlinks=true,
  linkcolor=blue,
  citecolor=blue,
  urlcolor=blue
}

\hyphenpenalty=950

\volume{0}
\issue{0}
\pubyear{2025}
\doi{0000}

\begin{document}

\begin{frontmatter}

\title{\Large Supplementary Materials for:}
\subtitle{\large Principles for Open Data Curation: A Case Study with the New York City 311 Service Request Data\\[2ex]}

\author[1]{\inits{D.}\fnms{David}~\snm{Tussey}}
\author[2]{\inits{J.}\fnms{Jun}~\snm{Yan}}

\address[1]{\institution{NYC Department of Information Technology and Telecommunications}, \cny{USA}}
\address[2]{Department of Statistics, \institution{University of Connecticut}, \cny{USA}}

\begin{keywords}
  \kwd{Data cleansing}
  \kwd{Data quality}
  \kwd{Data science}
  \kwd{Data Validation}
  \kwd{NYC Open Data}
\end{keywords}

\end{frontmatter}

%---------------------------------------------------------------
\section{Overview}

This document provides supplementary materials accompanying the article
\emph{Principles for Open Data Curation: A Case Study with the New York City 
311 Service Request Data}. It includes additional figures, tables, extended 
methodological details, data-processing steps, selected console output, 
and reproducibility information that support—but are not included 
in—the main manuscript.

The supplementary materials are intended to improve transparency,
facilitate reproducibility, and provide additional technical context
for readers interested in the data-cleaning and validation processes
applied in this study.

%---------------------------------------------------------------
\section{Supporting Charts}

\clearpage
\subsection{Yearly NYC 311 Service Request Summary Charts}

\begin{figure}[!htbp]
\centering
\includegraphics[width=0.85\textwidth]{annual_trend_with_projection_bar_chart.pdf}
\caption{Annual service request volume trend with 2025 projection.}
\label{fig:annual-trend-projection}
\end{figure}

\begin{figure}[!htbp]
\centering
\includegraphics[width=0.85\textwidth]{SR_by_agency_pareto_combo_chart.pdf}
\caption{Service request volume by agency (Pareto combined chart).}
\label{fig:sr-by-agency-pareto}
\end{figure}

\begin{figure}[!htbp]
\centering
\includegraphics[width=0.85\textwidth]{data_completeness_by_field_bar_chart.pdf}
\caption{Data completeness by field (bar chart).}
\label{fig:data-completeness-bar}
\end{figure}

\clearpage
\subsection{Latitude and Longitude Analysis}

\begin{figure}[!htbp]
\centering
\includegraphics[width=0.85\textwidth]{pareto_low_precision_latitude_by_agency.pdf}
\caption{Pareto chart of low-precision latitude values by agency.}
\label{fig:pareto-low-precision-latitude}
\end{figure}

\begin{figure}[!htbp]
\centering
\includegraphics[width=0.85\textwidth]{pareto_low_precision_longitude_by_agency.pdf}
\caption{Pareto chart of low-precision longitude values by agency.}
\label{fig:pareto-low-precision-longitude}
\end{figure}

\clearpage
\subsection{Duplicate Field Detection}

\begin{figure}[!htbp]
\centering
\includegraphics[width=0.85\textwidth]{pareto_street_field_non_matches_cross_street_1_vs_intersection_street_1.pdf}
\caption{Pareto chart of \texttt{cross\_street\_1} vs \texttt{intersection\_street\_1} non-matches by agency.}
\label{fig:pareto-street-nonmatch-1}
\end{figure}

\begin{figure}[!htbp]
\centering
\includegraphics[width=0.85\textwidth]{pareto_street_field_non_matches_cross_street_2_vs_intersection_street_2.pdf}
\caption{Pareto chart of \texttt{cross\_street\_2} vs \texttt{intersection\_street\_2} non-matches by agency.}
\label{fig:pareto-street-nonmatch-2}
\end{figure}

\begin{figure}[!htbp]
\centering
\includegraphics[width=0.85\textwidth]{pareto_street_field_non_matches_landmark_vs_street_name.pdf}
\caption{Pareto chart of \texttt{landmark} vs \texttt{street\_name} non-matches by agency.}
\label{fig:pareto-landmark-street-nonmatch}
\end{figure}

\begin{figure}[!htbp]
\centering
\includegraphics[width=0.85\textwidth]{pareto_nonmatches_taxi_company_borough_by_agency.pdf}
\caption{Pareto chart of \texttt{taxi\_company\_borough} non-matches by agency.}
\label{fig:pareto-taxi-borough-nonmatch}
\end{figure}


\clearpage
\subsection{Status and Closure Consistency Analysis}

\begin{figure}[!htbp]
\centering
\includegraphics[width=0.85\textwidth]{pareto_closed_status_missing_closed_date_by_agency.pdf}
\caption{Pareto chart of CLOSED status but missing \texttt{closed\_date} by agency.}
\label{fig:pareto-closed-status-missing-date}
\end{figure}

\begin{figure}[!htbp]
\centering
\includegraphics[width=0.85\textwidth]{pareto_nonclosed_status_with_closed_date_by_agency.pdf}
\caption{Pareto chart of non-CLOSED status but has \texttt{closed\_date} by agency.}
\label{fig:pareto-nonclosed-with-date}
\end{figure}


\clearpage
\subsection{Invalid Values Analysis}


\begin{figure}[!htbp]
\centering
\includegraphics[width=0.85\textwidth]{pareto_agency_community_board.pdf}
\caption{Pareto chart of invalid community board values by agency.}
\label{fig:pareto-community-board}
\end{figure}


\begin{figure}[!htbp]
\centering
\includegraphics[width=0.85\textwidth]{pareto_agency_incident_zip.pdf}
\caption{Pareto chart of invalid incident ZIP codes by agency.}
\label{fig:pareto-incident-zip}
\end{figure}


\clearpage
\subsection{Date Field Analysis}

% Yearly distribution figures - created_date
\begin{figure}[htbp]
\centering
\includegraphics[width=0.85\textwidth]{Yearly_Distribution - created_date.pdf}
\caption{Yearly distribution of NYC 311 service requests by created date, 2020--2024.}
\label{fig:yearly_created}
\end{figure}

% Yearly distribution figures - closed_date
\begin{figure}[htbp]
\centering
\includegraphics[width=0.85\textwidth]{Yearly_Distribution - closed_date.pdf}
\caption{Yearly distribution of NYC 311 service requests by closed date, all years.}
\label{fig:yearly_closed}
\end{figure}

% Yearly distribution figures - due_date
\begin{figure}[htbp]
\centering
\includegraphics[width=0.85\textwidth]{Yearly_Distribution - due_date.pdf}
\caption{Yearly distribution of NYC 311 service requests by due date, 2020--2024.}
\label{fig:yearly_due}
\end{figure}

% Yearly distribution figures - resolution_action_updated_date
\begin{figure}[htbp]
\centering
\includegraphics[width=0.85\textwidth]{Yearly_Distribution - resolution_action_updated_date.pdf}
\caption{Yearly distribution of NYC 311 service requests by resolution action updated date, 2020--2024.}
\label{fig:yearly_resolution}
\end{figure}


\clearpage
\subsection{Service Request (SR) Duration Analysis}

% Zero duration Pareto chart
\begin{figure}[htbp]
\centering
\includegraphics[width=0.85\textwidth]{zero_duration_pareto_combo_chart.pdf}
\caption{Pareto analysis of NYC 311 complaint types with zero-duration records.}
\label{fig:zero_duration_pareto}
\end{figure}


\begin{figure}[!htbp]
\centering
\includegraphics[width=0.85\textwidth]{boxplot_negative_days_by_agency.pdf}
\caption{Distribution of negative durations by agency.}
\label{fig:boxplot-negative-durations}
\end{figure}


\begin{figure}[!htbp]
\centering
\includegraphics[width=0.85\textwidth]{boxplot_negative_resolution_updates_by_agency.pdf}
\caption{Distribution of \texttt{resolution\_action\_updated\_date} before \texttt{created\_date} by agency.}
\label{fig:boxplot-negative-resolution-updates}
\end{figure}


\begin{figure}[!htbp]
\centering
\includegraphics[width=0.85\textwidth]{boxplot_near_zero_days_by_agency.pdf}
\caption{Distribution of near-zero duration records (2--28 seconds) by agency.}
\label{fig:boxplot-near-zero-durations}
\end{figure}

\begin{figure}[!htbp]
\centering
\includegraphics[width=0.85\textwidth]{boxplot_large_positive_days_by_agency.pdf}
\caption{Distribution of large positive durations (730--1{,}826 days) by agency.}
\label{fig:boxplot-large-positive-durations}
\end{figure}

\begin{figure}[!htbp]
\centering
\includegraphics[width=0.85\textwidth]{boxplot_extreme_positive_days_by_agency.pdf}
\caption{Distribution of extreme positive durations ($>$1{,}826 days) by agency.}
\label{fig:boxplot-extreme-positive-durations}
\end{figure}


\clearpage
\subsection{Temporal Pattern Analysis Visualizations}


\begin{figure}[!htbp]
\centering
\includegraphics[width=0.85\textwidth]{midnight_created_dates_pareto_combo_chart.pdf}
\caption{Combined Pareto analysis of midnight \texttt{created\_date} anomalies.}
\label{fig:midnight-created-pareto-combo}
\end{figure}


\begin{figure}[!htbp]
\centering
\includegraphics[width=0.85\textwidth]{noon_created_dates_pareto_combo_chart.pdf}
\caption{Combined Pareto analysis of noon \texttt{created\_date} anomalies.}
\label{fig:noon-created-pareto-combo}
\end{figure}

\clearpage
\subsubsection{Created Date Patterns}

\begin{figure}[!htbp]
\centering
\includegraphics[width=0.85\textwidth]{created_hour_distribution_working_order.pdf}
\caption{Hourly distribution of \texttt{created\_date} timestamps (working day order).}
\label{fig:created-hour-distribution}
\end{figure}

\begin{figure}[!htbp]
\centering
\includegraphics[width=0.85\textwidth]{created_top_of_hour_distribution.pdf}
\caption{Top-of-hour distribution for \texttt{created\_date} timestamps.}
\label{fig:created-top-of-hour}
\end{figure}

\begin{figure}[!htbp]
\centering
\includegraphics[width=0.85\textwidth]{created_exact_midnight_cy_distribution.pdf}
\caption{Calendar year distribution of midnight \texttt{created\_date} entries.}
\label{fig:created-exact-midnight-cy}
\end{figure}

\begin{figure}[!htbp]
\centering
\includegraphics[width=0.85\textwidth]{created_exact_noon_cy_distribution.pdf}
\caption{Calendar year distribution of noon \texttt{created\_date} entries.}
\label{fig:created-exact-noon-cy}
\end{figure}

\begin{figure}[!htbp]
\centering
\includegraphics[width=0.85\textwidth]{pareto_created_exact_midnight_agency.pdf}
\caption{Pareto chart of midnight \texttt{created\_date} entries by agency.}
\label{fig:pareto-created-midnight}
\end{figure}

\begin{figure}[!htbp]
\centering
\includegraphics[width=0.85\textwidth]{pareto_created_exact_noon_agency.pdf}
\caption{Pareto chart of noon \texttt{created\_date} entries by agency.}
\label{fig:pareto-created-noon}
\end{figure}


\clearpage
\subsubsection{Closed Date Patterns}

\begin{figure}[!htbp]
\centering
\includegraphics[width=0.85\textwidth]{closed_hour_distribution_working_order.pdf}
\caption{Hourly distribution of \texttt{closed\_date} timestamps (working day order).}
\label{fig:closed-hour-distribution}
\end{figure}

\begin{figure}[!htbp]
\centering
\includegraphics[width=0.85\textwidth]{closed_top_of_hour_distribution.pdf}
\caption{Top-of-hour distribution for \texttt{closed\_date} timestamps.}
\label{fig:closed-top-of-hour}
\end{figure}

\begin{figure}[!htbp]
\centering
\includegraphics[width=0.85\textwidth]{closed_exact_midnight_cy_distribution.pdf}
\caption{Calendar year distribution of midnight \texttt{closed\_date} entries.}
\label{fig:closed-exact-midnight-cy}
\end{figure}

\begin{figure}[!htbp]
\centering
\includegraphics[width=0.85\textwidth]{closed_exact_noon_cy_distribution.pdf}
\caption{Calendar year distribution of noon \texttt{closed\_date} entries.}
\label{fig:closed-exact-noon-cy}
\end{figure}

\begin{figure}[!htbp]
\centering
\includegraphics[width=0.85\textwidth]{pareto_closed_exact_midnight_agency.pdf}
\caption{Pareto chart of midnight \texttt{closed\_date} entries by agency.}
\label{fig:pareto-closed-midnight}
\end{figure}

\begin{figure}[!htbp]
\centering
\includegraphics[width=0.85\textwidth]{pareto_closed_exact_noon_agency.pdf}
\caption{Pareto chart of noon \texttt{closed\_date} entries by agency.}
\label{fig:pareto-closed-noon}
\end{figure}

\clearpage
\subsubsection{Resolution Action Updated Date Patterns}

\begin{figure}[!htbp]
\centering
\includegraphics[width=0.85\textwidth]{resolutionactionupdated_hour_distribution_working_order.pdf}
\caption{Hourly distribution of \texttt{resolution\_action\_updated\_date} timestamps (working day order).}
\label{fig:resolution-hour-distribution}
\end{figure}

\begin{figure}[!htbp]
\centering
\includegraphics[width=0.85\textwidth]{resolutionactionupdated_top_of_hour_distribution.pdf}
\caption{Top-of-hour distribution for \texttt{resolution\_action\_updated\_date} timestamps.}
\label{fig:resolution-top-of-hour}
\end{figure}

\begin{figure}[!htbp]
\centering
\includegraphics[width=0.85\textwidth]{resolutionactionupdated_exact_midnight_cy_distribution.pdf}
\caption{Calendar year distribution of midnight \texttt{resolution\_action\_updated\_date} entries.}
\label{fig:resolution-exact-midnight-cy}
\end{figure}

\begin{figure}[!htbp]
\centering
\includegraphics[width=0.85\textwidth]{resolutionactionupdated_exact_noon_cy_distribution.pdf}
\caption{Calendar year distribution of noon \texttt{resolution\_action\_updated\_date} entries.}
\label{fig:resolution-exact-noon-cy}
\end{figure}

\begin{figure}[!htbp]
\centering
\includegraphics[width=0.85\textwidth]{pareto_resolutionactionupdated_exact_midnight_agency.pdf}
\caption{Pareto chart of midnight \texttt{resolution\_action\_updated\_date} entries by agency.}
\label{fig:pareto-resolution-midnight}
\end{figure}

\begin{figure}[!htbp]
\centering
\includegraphics[width=0.85\textwidth]{pareto_resolutionactionupdated_exact_noon_agency.pdf}
\caption{Pareto chart of noon \texttt{resolution\_action\_updated\_date} entries by agency.}
\label{fig:pareto-resolution-noon}
\end{figure}

\begin{figure}[!htbp]
\centering
\includegraphics[width=0.85\textwidth]{midnight_resolution_action_updated_dates_pareto_combo_chart.pdf}
\caption{Combined Pareto analysis of midnight \texttt{resolution\_action\_updated\_date} anomalies.}
\label{fig:midnight-resolution-pareto-combo}
\end{figure}

\begin{figure}[!htbp]
\centering
\includegraphics[width=0.85\textwidth]{noon_resolution_action_updated_dates_pareto_combo_chart.pdf}
\caption{Combined Pareto analysis of noon \texttt{resolution\_action\_updated\_date} anomalies.}
\label{fig:noon-resolution-pareto-combo}
\end{figure}

\begin{figure}[!htbp]
\centering
\includegraphics[width=0.85\textwidth]{histogram_resolution_updates_before_created.pdf}
\caption{Histogram of \texttt{resolution\_action\_updated\_date} before \texttt{created\_date}.}
\label{fig:histogram-resolution-before-created}
\end{figure}

\begin{figure}[!htbp]
\centering
\includegraphics[width=0.85\textwidth]{pareto_negative_resolution_updates_by_agency.pdf}
\caption{Pareto chart of negative resolution update durations by agency.}
\label{fig:pareto-negative-resolution-updates}
\end{figure}

\begin{figure}[!htbp]
\centering
\includegraphics[width=0.85\textwidth]{resolution_action_updated_dates_before_created_date_pareto_combo_chart.pdf}
\caption{Pareto chart showing resolution updates that occurred before request creation.}
\label{fig:resolution-before-created-pareto-combo}
\end{figure}
%
%\clearpage
%\subsection{Daylight Saving Time Analysis}
%
%\begin{figure}[!htbp]
%\centering
%\includegraphics[width=0.85\textwidth]{dst_fallback_negative_duration_by_date.pdf}
%\caption{DST fall-back negative duration anomalies by date.}
%\label{fig:dst-fallback-negative}
%\end{figure}
%
%\begin{figure}[!htbp]
%\centering
%\includegraphics[width=0.85\textwidth]{dst_springforward_overlap_by_date.pdf}
%\caption{DST spring-forward overlap anomalies by date.}
%\label{fig:dst-springforward-overlap}
%\end{figure}
%

\clearpage
\subsection{Duration Analysis Visualizations}

\begin{figure}[!htbp]
\centering
\includegraphics[width=0.85\textwidth]{duration_histogram.pdf}
\caption{Suspicious/Questionable duration histogram.}
\label{fig:duration-histogram}
\end{figure}

\begin{figure}[!htbp]
\centering
\includegraphics[width=0.85\textwidth]{duration_histogram_cumulative.pdf}
\caption{Cumulative Suspicious duration distribution.}
\label{fig:duration-histogram-cumulative}
\end{figure}

\begin{figure}[!htbp]
\centering
\includegraphics[width=0.85\textwidth]{positive_duration_histogram.pdf}
\caption{Positive duration histogram.}
\label{fig:positive-duration-histogram}
\end{figure}

\begin{figure}[!htbp]
\centering
\includegraphics[width=0.85\textwidth]{negative_duration_histogram.pdf}
\caption{Negative duration histogram.}
\label{fig:negative-duration-histogram}
\end{figure}
%
%\begin{figure}[!htbp]
%\centering
%\includegraphics[width=0.85\textwidth]{skewed_duration_analysis_comparison.pdf}
%\caption{Skewed duration analysis comparison.}
%\label{fig:skewed-duration-comparison}
%\end{figure}

\begin{figure}[!htbp]
\centering
\includegraphics[width=0.85\textwidth]{skewed_duration_analysis_methods.pdf}
\caption{Outlier detection methods for skewed duration data.}
\label{fig:skewed-duration-methods}
\end{figure}

\clearpage
\subsubsection{Negative Duration Visualizations}

\begin{figure}[!htbp]
\centering
\includegraphics[width=0.85\textwidth]{negative_duration_pareto_combo_chart.pdf}
\caption{Combined Pareto analysis of negative durations by agency.}
\label{fig:negative-duration-pareto-combo}
\end{figure}

\begin{figure}[!htbp]
\centering
\includegraphics[width=0.85\textwidth]{negative_duration_SR_violin.pdf}
\caption{Violin plot of negative durations by agency.}
\label{fig:negative-duration-violin}
\end{figure}

\begin{figure}[!htbp]
\centering
\includegraphics[width=0.85\textwidth]{violin_boxplot_negative_days_by_agency.pdf}
\caption{Combined violin and boxplot for negative durations by agency.}
\label{fig:violin-boxplot-negative}
\end{figure}

\begin{figure}[!htbp]
\centering
\includegraphics[width=0.85\textwidth]{extreme_negative_duration_pareto_combo_chart.pdf}
\caption{Combined Pareto analysis of extreme negative durations by agency.}
\label{fig:extreme-negative-pareto-combo}
\end{figure}


\clearpage
\subsubsection{Zero and Near-Zero Duration Visualizations}

\begin{figure}[!htbp]
\centering
\includegraphics[width=0.85\textwidth]{zero_duration_pareto_combo_chart.pdf}
\caption{Combined Pareto analysis of zero-duration records by agency.}
\label{fig:zero-duration-pareto-combo}
\end{figure}


\begin{figure}[!htbp]
\centering
\includegraphics[width=0.85\textwidth]{near_zero_duration_pareto_combo_chart.pdf}
\caption{Combined Pareto analysis of near-zero duration records (2--28 seconds) by agency.}
\label{fig:near-zero-pareto-combo}
\end{figure}

\begin{figure}[!htbp]
\centering
\includegraphics[width=0.85\textwidth]{violin_boxplot_near_zero_days_by_agency.pdf}
\caption{Combined violin and boxplot for near-zero duration records by agency.}
\label{fig:violin-boxplot-near-zero}
\end{figure}

\clearpage
\subsubsection{Large Positive Duration Visualizations}

\begin{figure}[!htbp]
\centering
\includegraphics[width=0.85\textwidth]{large_positive_duration_pareto_combo_chart.pdf}
\caption{Combined Pareto analysis of large positive durations (730--1{,}826 days) by agency.}
\label{fig:large-positive-pareto-combo}
\end{figure}

\begin{figure}[!htbp]
\centering
\includegraphics[width=0.85\textwidth]{violin_boxplot_large_positive_days_by_agency.pdf}
\caption{Combined violin and boxplot for large positive durations by agency.}
\label{fig:violin-boxplot-large-positive}
\end{figure}

\begin{figure}[!htbp]
\centering
\includegraphics[width=0.85\textwidth]{extreme_positive_duration_pareto_combo_chart.pdf}
\caption{Combined Pareto analysis of extreme positive durations ($>$1{,}826 days) by agency.}
\label{fig:extreme-positive-pareto-combo}
\end{figure}

\begin{figure}[!htbp]
\centering
\includegraphics[width=0.85\textwidth]{violin_boxplot_extreme_positive_days_by_agency.pdf}
\caption{Combined violin and boxplot for extreme positive durations by agency.}
\label{fig:violin-boxplot-extreme-positive}
\end{figure}


\begin{figure}[!htbp]
\centering
\includegraphics[width=0.85\textwidth]{post_closed_updates_histogram.pdf}
\caption{Histogram of updates after closure.}
\label{fig:post-closed-updates-histogram}
\end{figure}

\begin{figure}[!htbp]
\centering
\includegraphics[width=0.85\textwidth]{post_closed_updates_pareto.pdf}
\caption{Pareto chart of updates after closure by agency.}
\label{fig:post-closed-updates-pareto}
\end{figure}

\begin{figure}[!htbp]
\centering
\includegraphics[width=0.85\textwidth]{post_closed_updates_boxplot.pdf}
\caption{Boxplot of updates after closure by agency.}
\label{fig:post-closed-updates-boxplot}
\end{figure}
%
%\begin{figure}[!htbp]
%\centering
%\includegraphics[width=0.85\textwidth]{post_closed_updates_violin_boxplot.pdf}
%\caption{Combined violin and boxplot for updates after closure by agency.}
%\label{fig:post-closed-updates-violin-boxplot}
%\end{figure}

\begin{figure}[!htbp]
\centering
\includegraphics[width=0.85\textwidth]{post_closed_resolution_violin.pdf}
\caption{Violin plot of resolution updates after closure by agency.}
\label{fig:post-closed-resolution-violin}
\end{figure}

\clearpage
\subsection{Miscellaneous Validation Charts}






%---------------------------------------------------------------
\section{Supplemental Tables}

% Start Supplement tables/figures at S1
\setcounter{table}{0}
\setcounter{figure}{0}

This section presents detailed tabular outputs generated during the data
validation and cleansing process. Tables are numbered sequentially
(Table~S1, Table~S2, \ldots) and are referenced independently of the main manuscript.

%---------------------------------------------------------------
\paragraph{Execution context.}
Execution begins at: 2025-11-29 14:22:40.

%---------------------------------------------------------------
\subsection{Column-level completeness (filled counts and \% complete)}

\begin{table}[!htbp]
\centering
\begin{tabular}{lrr}
\toprule
Field & Filled (N) & Complete (\%) \\
\midrule
taxi\_company\_borough         &        8{,}564 &   0.0535 \\
road\_ramp                     &       34{,}117 &   0.2130 \\
due\_date                      &       52{,}346 &   0.3268 \\
bridge\_highway\_direction     &       55{,}072 &   0.3438 \\
bridge\_highway\_segment       &      104{,}298 &   0.6512 \\
bridge\_highway\_name          &      104{,}523 &   0.6526 \\
taxi\_pick\_up\_location       &      136{,}863 &   0.8545 \\
vehicle\_type                  &      181{,}533 &   1.1334 \\
facility\_type                 &      211{,}240 &   1.3189 \\
landmark                       &    9{,}058{,}628 &  56.5574 \\
intersection\_street\_1        &   10{,}251{,}224 &  64.0033 \\
intersection\_street\_2        &   10{,}257{,}371 &  64.0417 \\
cross\_street\_2               &   11{,}693{,}685 &  73.0093 \\
cross\_street\_1               &   11{,}696{,}472 &  73.0267 \\
address\_type                  &   13{,}246{,}948 &  82.7071 \\
location\_type                 &   13{,}823{,}312 &  86.3056 \\
bbl                            &   14{,}128{,}219 &  88.2093 \\
city                           &   15{,}156{,}737 &  94.6308 \\
resolution\_description        &   15{,}274{,}251 &  95.3645 \\
street\_name                   &   15{,}317{,}675 &  95.6356 \\
incident\_address              &   15{,}318{,}222 &  95.6391 \\
descriptor                     &   15{,}519{,}553 &  96.8961 \\
latitude                       &   15{,}713{,}333 &  98.1059 \\
longitude                      &   15{,}713{,}333 &  98.1059 \\
location                       &   15{,}713{,}333 &  98.1059 \\
x\_coordinate\_state\_plane     &   15{,}713{,}468 &  98.1068 \\
y\_coordinate\_state\_plane     &   15{,}714{,}705 &  98.1145 \\
incident\_zip                  &   15{,}757{,}103 &  98.3792 \\
closed\_date                   &   15{,}785{,}892 &  98.5590 \\
resolution\_action\_updated\_date & 15{,}911{,}615 &  99.3439 \\
community\_board               &   15{,}978{,}265 &  99.7600 \\
borough                        &   15{,}978{,}265 &  99.7600 \\
park\_borough                  &   15{,}978{,}265 &  99.7600 \\
park\_facility\_name           &   16{,}005{,}793 &  99.9319 \\
unique\_key                    &   16{,}016{,}700 & 100.0000 \\
created\_date                  &   16{,}016{,}700 & 100.0000 \\
agency                         &   16{,}016{,}700 & 100.0000 \\
agency\_name                   &   16{,}016{,}700 & 100.0000 \\
complaint\_type                &   16{,}016{,}700 & 100.0000 \\
status                         &   16{,}016{,}700 & 100.0000 \\
open\_data\_channel\_type      &   16{,}016{,}700 & 100.0000 \\
\bottomrule
\end{tabular}
\caption{Number and percent of complete (non-missing) entries per column.}
\label{tab:column-completeness}
\end{table}

\paragraph{Interpretation.}
Core administrative fields are fully populated.
Examples include \texttt{unique\_key}, \texttt{created\_date},
\texttt{agency}, and \texttt{complaint\_type}.

%---------------------------------------------------------------
\subsection{Overall dataset completeness}

\begin{table}[!htbp]
\centering
\begin{tabular}{lr}
\toprule
Metric & Value \\
\midrule
Total cells          & 656{,}684{,}700 \\
Total NA/missing     & 182{,}973{,}577 \\
Total filled         & 473{,}711{,}123 \\
Percent missing (\%) & 27.86 \\
Percent complete (\%) & 72.14 \\
\bottomrule
\end{tabular}
\caption{Overall completeness metrics across all fields and records.}
\label{tab:overall-completeness}
\end{table}

\paragraph{Interpretation.}
The overall completeness rate is moderately high, but missingness is concentrated in a
subset of fields; field-specific validation and cleaning rules are therefore more appropriate
than uniform assumptions.

%---------------------------------------------------------------
\subsection{Field usage by agency}

The complete field-by-agency usage matrix is provided as a machine-readable
comma-separated values (CSV) file due to its size and dimensionality.
The table records the number of non-missing observations for each data
field, stratified by reporting agency, across the full NYC~311 dataset
used in this study.

The CSV file is available at:
\begin{quote}
\url{https://figshare.com/s/9f878b50687c9e4c540a}
\end{quote}

Each row corresponds to a single data field, and each column corresponds
to a reporting agency. Cell values indicate the count of records in which
the given field is populated for the specified agency. The final column
contains the total count across all agencies.

\paragraph{Interpretation.}
Field usage varies substantially by agency, reflecting differences in
operational responsibilities, reporting practices, and data-collection
requirements. Core administrative fields (e.g., \texttt{unique\_key},
\texttt{created\_date}, \texttt{agency}, and \texttt{complaint\_type})
are fully populated across all agencies, whereas location-specific,
infrastructure-related, and program-specific fields are populated only
by agencies for which they are operationally relevant.

%---------------------------------------------------------------
\subsection{Street pair analysis: (cross\_street\_1 vs intersection\_street\_1)}

\begin{table}[!htbp]
\centering
\begin{tabular}{lrr}
\toprule
Category & N & Percent (\%) \\
\midrule
Total records & 16{,}016{,}700 & {} \\
\addlinespace
Matches (total) & 13{,}388{,}498 & 83.6 \\
\quad Both fields non-blank & 9{,}601{,}460 & 59.9 \\
\quad Both fields blank & 3{,}787{,}038 & 23.6 \\
\addlinespace
Non-matches (total) & 2{,}628{,}202 & 16.4 \\
\quad Non-matches both non-blank & 116{,}574 & 0.7 \\
\quad Non-matches cross\_street\_1 blank & 533{,}190 & 3.3 \\
\quad Non-matches intersection\_street\_1 blank & 1{,}978{,}438 & 12.4 \\
\bottomrule
\end{tabular}
\caption{Matching outcomes for \texttt{cross\_street\_1} vs \texttt{intersection\_street\_1}.}
\label{tab:streetpair1-match-summary}
\end{table}

\paragraph{Interpretation.}
True disagreements with both fields populated are rare. Most non-matches are caused
by missingness in one field (especially \texttt{intersection\_street\_1}).

%---------------------------------------------------------------
\subsection{Street pair analysis 1 --- non-matches by agency (Pareto)}

\begin{table}[!htbp]
\centering
\begin{tabular}{lrrr}
\toprule
Agency & N & Pct & Cum.\ Pct \\
\midrule
DSNY  & 1{,}113{,}755 & 0.42 & 0.42 \\
DEP   &   830{,}484 & 0.32 & 0.74 \\
DOT   &   612{,}283 & 0.23 & 0.97 \\
DOHMH &    37{,}739 & 0.01 & 0.99 \\
DHS   &    33{,}940 & 0.01 & 1.00 \\
DOB   &         1 & 0.00 & 1.00 \\
\bottomrule
\end{tabular}
\caption{Pareto summary of \texttt{cross\_street\_1} vs \texttt{intersection\_street\_1} non-matches by agency.}
\label{tab:streetpair1-nonmatch-pareto}
\end{table}

\paragraph{Interpretation.}
Non-matches are highly concentrated in DSNY, DEP, and DOT, suggesting that
agency-specific field population practices drive the majority of mismatch volume.

%---------------------------------------------------------------
\subsection{Street pair analysis: (cross\_street\_2 vs intersection\_street\_2)}

\begin{table}[!htbp]
\centering
\begin{tabular}{lrr}
\toprule
Category & N & Percent (\%) \\
\midrule
Total records & 16{,}016{,}700 & {} \\
\addlinespace
Matches (total) & 13{,}391{,}661 & 83.6 \\
\quad Both fields non-blank & 9{,}603{,}211 & 60.0 \\
\quad Both fields blank & 3{,}788{,}450 & 23.7 \\
\addlinespace
Non-matches (total) & 2{,}625{,}039 & 16.4 \\
\quad Non-matches both non-blank & 119{,}595 & 0.7 \\
\quad Non-matches cross\_street\_2 blank & 534{,}565 & 3.3 \\
\quad Non-matches intersection\_street\_2 blank & 1{,}970{,}879 & 12.3 \\
\bottomrule
\end{tabular}
\caption{Matching outcomes for \texttt{cross\_street\_2} vs \texttt{intersection\_street\_2}.}
\label{tab:streetpair2-match-summary}
\end{table}

\paragraph{Interpretation.}
Results mirror those of Street Pair 1, again indicating that missing values—not
conflicting values—account for most mismatches.

%---------------------------------------------------------------
\subsection{Street pair analysis 2 --- non-matches by agency (Pareto)}

\begin{table}[!htbp]
\centering
\begin{tabular}{lrrr}
\toprule
Agency & N & Pct & Cum.\ Pct \\
\midrule
DSNY  & 1{,}112{,}745 & 0.42 & 0.42 \\
DEP   &   830{,}209 & 0.32 & 0.74 \\
DOT   &   604{,}379 & 0.23 & 0.97 \\
DOHMH &    39{,}133 & 0.01 & 0.99 \\
DHS   &    38{,}572 & 0.01 & 1.00 \\
DOB   &         1 & 0.00 & 1.00 \\
\bottomrule
\end{tabular}
\caption{Pareto summary of \texttt{cross\_street\_2} vs \texttt{intersection\_street\_2} non-matches by agency.}
\label{tab:streetpair2-nonmatch-pareto}
\end{table}

\paragraph{Interpretation.}
As in Street Pair 1, DSNY and DEP dominate non-match volume, consistent
with differential completeness/usage of intersection fields.

%---------------------------------------------------------------
\subsection{Street pair analysis 3 (landmark vs street\_name) --- match summary}

\begin{table}[!htbp]
\centering
\begin{tabular}{lrr}
\toprule
Category & N & Percent (\%) \\
\midrule
Total records & 16{,}016{,}700 & {} \\
\addlinespace
Matches (total) & 9{,}558{,}195 & 59.7 \\
\quad Both fields non-blank & 8{,}860{,}665 & 55.3 \\
\quad Both fields blank & 697{,}530 & 4.4 \\
\addlinespace
Non-matches (total) & 6{,}458{,}505 & 40.3 \\
\quad Non-matches both non-blank & 196{,}468 & 1.2 \\
\quad Non-matches landmark blank & 6{,}260{,}542 & 39.1 \\
\quad Non-matches street\_name blank & 1{,}495 & 0.0 \\
\bottomrule
\end{tabular}
\caption{Matching outcomes for \texttt{landmark} vs \texttt{street\_name}.}
\label{tab:landmark-streetname-match-summary}
\end{table}

\paragraph{Interpretation.}
The lower match rate is driven almost entirely by missing \texttt{landmark} values, indicating limited applicability and inconsistent field population rather than widespread contradictory location content.

%---------------------------------------------------------------
\subsection{Street pair analysis 3 --- non-matches by agency (Pareto)}

\begin{table}[!htbp]
\centering
\begin{tabular}{lrrr}
\toprule
Agency & N & Pct & Cum.\ Pct \\
\midrule
HPD   & 3{,}054{,}700 & 0.47 & 0.47 \\
DSNY  & 1{,}226{,}834 & 0.19 & 0.66 \\
DEP   &   736{,}128 & 0.11 & 0.78 \\
NYPD  &   512{,}760 & 0.08 & 0.86 \\
DOB   &   470{,}236 & 0.07 & 0.93 \\
DOT   &   294{,}229 & 0.05 & 0.97 \\
DPR   &    55{,}885 & 0.01 & 0.98 \\
TLC   &    34{,}248 & 0.01 & 0.99 \\
DHS   &    32{,}074 & 0.00 & 0.99 \\
DOHMH &    25{,}373 & 0.00 & 1.00 \\
DCWP  &    11{,}191 & 0.00 & 1.00 \\
EDC   &     1{,}722 & 0.00 & 1.00 \\
DFTA  &     1{,}495 & 0.00 & 1.00 \\
OSE   &     1{,}288 & 0.00 & 1.00 \\
DOE   &       191 & 0.00 & 1.00 \\
DOITT &       109 & 0.00 & 1.00 \\
OTI   &        41 & 0.00 & 1.00 \\
3-1-1 &         1 & 0.00 & 1.00 \\
\bottomrule
\end{tabular}
\caption{Pareto summary of \texttt{landmark} vs \texttt{street\_name} non-matches by agency.}
\label{tab:landmark-streetname-nonmatch-pareto}
\end{table}

\paragraph{Interpretation.}
Non-matches are dominated by HPD and DSNY, consistent with
large volumes and comparatively low \texttt{landmark} population. This points
to field-usage conventions as the main driver rather than semantic inconsistency.

%---------------------------------------------------------------
\subsection{Match rates across all street pairs (summary)}

\begin{table}[!htbp]
\centering
\begin{tabular}{lr}
\toprule
Street Pair & Match Rate (\%) \\
\midrule
cross\_street\_1 vs intersection\_street\_1 & 83.5909 \\
cross\_street\_2 vs intersection\_street\_2 & 83.6106 \\
landmark vs street\_name & 59.6764 \\
\bottomrule
\end{tabular}
\caption{Summary match rates across all street pair analyses. Total analyses completed: 3.}
\label{tab:streetpair-matchrates-summary}
\end{table}

\paragraph{Interpretation.}
Cross-street and intersection-street pairs show high consistency
when both are populated. The landmark field behaves differently,
with substantially lower effective coverage.

%---------------------------------------------------------------
\subsection{Multi-year statistics (Actuals + Projected)}

\begin{table}[!htbp]
\centering
\begin{tabular}{lr}
\toprule
Statistic & Value \\
\midrule
Years covered & 2020--2025 \\
Total records & 19{,}525{,}499 \\
Yearly mean & 3{,}254{,}250 \\
Yearly median & 3{,}223{,}236 \\
Std.\ dev.\ & 206{,}378 \\
Max year & 2025 (3{,}508{,}799) \\
Min year & 2020 (2{,}942{,}064) \\
Growth (\%) & 19.3 \\
Busiest month & 2020-08 (348{,}463) \\
Least busy month & 2020-04 (159{,}115) \\
Busiest day & 2020-08-04 (24{,}415) \\
Least busy day & 2020-03-29 (3{,}785) \\
\bottomrule
\end{tabular}
\caption{Multi-year request-volume statistics for 2020--2025 (including projected 2025 total).}
\label{tab:multiyear-stats-2020-2025}
\end{table}

\paragraph{Interpretation.}
Volume increases over the period, with a clear low point in
early 2020 and a later peak in 2025. These aggregates help
contextualize agency and field-coverage patterns by overall workload.

%---------------------------------------------------------------
\subsection{Year-by-year service request counts}

\begin{table}[!htbp]
\centering
\begin{tabular}{lr}
\toprule
Year & Records (N) \\
\midrule
2020 & 2{,}942{,}064 \\
2021 & 3{,}220{,}915 \\
2022 & 3{,}169{,}844 \\
2023 & 3{,}225{,}557 \\
2024 & 3{,}458{,}320 \\
2025 & 3{,}508{,}799 (Projected) \\
\bottomrule
\end{tabular}
\caption{Year-by-year service request counts.}
\label{tab:yearly-request-counts}
\end{table}

\paragraph{Interpretation.}
Counts are relatively stable from 2021--2023, then rise in 2024
and remain elevated into the projected 2025 total.

%---------------------------------------------------------------
\subsection{Agency volume distribution (Pareto)}

\begin{table}[!htbp]
\centering
\begin{tabular}{lrrr}
\toprule
Agency & N & Pct & Cum.\ Pct \\
\midrule
NYPD  & 6{,}905{,}945 & 0.43 & 0.43 \\
HPD   & 3{,}067{,}970 & 0.19 & 0.62 \\
DSNY  & 1{,}923{,}489 & 0.12 & 0.74 \\
DOT   & 1{,}078{,}415 & 0.07 & 0.81 \\
DEP   &   834{,}539 & 0.05 & 0.86 \\
DPR   &   636{,}299 & 0.04 & 0.90 \\
DOB   &   470{,}508 & 0.03 & 0.93 \\
DOHMH &   397{,}968 & 0.02 & 0.96 \\
DHS   &   211{,}486 & 0.01 & 0.97 \\
EDC   &   150{,}305 & 0.01 & 0.98 \\
TLC   &   138{,}934 & 0.01 & 0.99 \\
DCWP  &   119{,}727 & 0.01 & 0.99 \\
OSE   &    70{,}955 & 0.00 & 1.00 \\
DOE   &     6{,}380 & 0.00 & 1.00 \\
DFTA  &     2{,}605 & 0.00 & 1.00 \\
DOITT &       669 & 0.00 & 1.00 \\
OTI   &       505 & 0.00 & 1.00 \\
3-1-1 &         1 & 0.00 & 1.00 \\
\bottomrule
\end{tabular}
\caption{Pareto summary of service request volume by agency.}
\label{tab:agency-volume-pareto}
\end{table}

\paragraph{Interpretation.}
A small number of agencies account for the majority of service requests,
with NYPD and HPD comprising the largest shares. This concentration
influences where data-quality interventions will yield the greatest impact.

%---------------------------------------------------------------
\subsection{Dataset dimensions and coverage window}

\begin{table}[!htbp]
\centering
\begin{tabular}{lr}
\toprule
Attribute & Value \\
\midrule
Rows in the 311 SR dataset & 16{,}016{,}700 \\
Columns in the 311 SR dataset & 41 \\
Agencies represented & 18 \\
Coverage window & 2020-01-01 00:00:00 through 2024-12-31 23:59:38 \\
\bottomrule
\end{tabular}
\caption{Basic dataset dimensions and temporal coverage as reported by the program output.}
\label{tab:dataset-dimensions-coverage}
\end{table}

\paragraph{Interpretation.}
These metadata provide context for the completeness and consistency
checks reported elsewhere in the Supplement.

%---------------------------------------------------------------
\subsection{Service requests by status (including NA if present)}

\begin{table}[!htbp]
\centering
\begin{tabular}{lrrr}
\toprule
Status & Count (N) & Percent (\%) & Cumulative (\%) \\
\midrule
CLOSED      & 15{,}774{,}141 & 98.49 & 98.49 \\
IN PROGRESS &   149{,}908 & 0.94 & 99.42 \\
PENDING     &    54{,}332 & 0.34 & 99.76 \\
OPEN        &    18{,}312 & 0.11 & 99.88 \\
ASSIGNED    &    13{,}126 & 0.08 & 99.96 \\
STARTED     &     4{,}294 & 0.03 & 99.98 \\
UNSPECIFIED &     2{,}586 & 0.02 & 100.00 \\
CANCEL      &         1 & 0.00 & 100.00 \\
\bottomrule
\end{tabular}
\caption{Distribution of service requests by status.}
\label{tab:status-distribution}
\end{table}

\paragraph{Interpretation.}
The overwhelming prevalence of \texttt{CLOSED} requests supports
analyses dependent on closure fields (e.g., durations). The small
fraction of active statuses should be treated explicitly in workflows
that assume populated closure timestamps.

%---------------------------------------------------------------
\subsection{Non-compliant \texttt{incident\_zip} values by reason (format-only; NA excluded)}

\begin{table}[!htbp]
\centering
\begin{tabular}{lr}
\toprule
Reason & N \\
\midrule
wrong\_length & 12 \\
non\_digit    & 2 \\
\bottomrule
\end{tabular}
\caption{Counts of non-compliant \texttt{incident\_zip} values by validation reason (NA excluded). Total non-compliant values: 14.}
\label{tab:zip-noncompliant-by-reason}
\end{table}

\paragraph{Interpretation.}
Format violations are rare, suggesting strong constraints on ZIP entry. These
cases remain valuable as test vectors for automated validation and correction logic.

%---------------------------------------------------------------
\subsection{Examples of non-compliant \texttt{incident\_zip} values (first 10 shown in output)}

\begin{table}[!htbp]
\centering
\begin{tabular}{rlll r}
\toprule
Unique Key & Agency & Incident ZIP & Reason & ZIP Length \\
\midrule
63{,}528{,}478 & DCWP & Unkno & non\_digit & 5 \\
63{,}119{,}597 & DCWP & 1172  & wrong\_length & 4 \\
61{,}958{,}164 & DSNY & 0000  & wrong\_length & 4 \\
61{,}310{,}558 & DCWP & 3361  & wrong\_length & 4 \\
61{,}123{,}976 & DCWP & 1155  & wrong\_length & 4 \\
52{,}003{,}981 & DCWP & 1055O & non\_digit & 5 \\
51{,}935{,}747 & DSNY & 0     & wrong\_length & 1 \\
51{,}936{,}203 & DSNY & 0     & wrong\_length & 1 \\
51{,}936{,}138 & DSNY & 0     & wrong\_length & 1 \\
51{,}646{,}769 & DSNY & 0     & wrong\_length & 1 \\
\bottomrule
\end{tabular}
\caption{Example records with non-compliant ZIP formats (as printed by the program output).}
\label{tab:zip-noncompliant-examples}
\end{table}

\paragraph{Interpretation.}
The examples show two dominant patterns: short strings such as \texttt{0} and
four-digit codes. Occasional alphanumeric strings (e.g., \texttt{1055O})
indicate character-entry contamination.

%---------------------------------------------------------------
\subsection{Non-compliant \texttt{incident\_zip} values by agency (top 10 as printed)}

\begin{table}[!htbp]
\centering
\begin{tabular}{lr}
\toprule
Agency & N \\
\midrule
DSNY & 9 \\
DCWP & 5 \\
\bottomrule
\end{tabular}
\caption{Counts of non-compliant ZIP formats by agency (top agencies shown in output).}
\label{tab:zip-noncompliant-by-agency}
\end{table}

\paragraph{Interpretation.}
Even though the counts are small, the clustering within DSNY and DCWP
supports agency-targeted validation checks at ingestion.

% =========================================================
\subsection{Type-only numeric checks (selected fields)}

\begin{table}[!htbp]
\centering
\begin{tabular}{lr}
\toprule
Column & \texttt{is\_numeric} \\
\midrule
\texttt{x\_coordinate\_state\_plane} & TRUE \\
\texttt{y\_coordinate\_state\_plane} & TRUE \\
\texttt{latitude} & TRUE \\
\texttt{longitude} & TRUE \\
\bottomrule
\end{tabular}
\caption{Type-only numeric checks for selected location fields.}
\label{tab:type-only-numeric-checks}
\end{table}

% =========================================================
\subsection{Latitude precision summary (decimal places)}

\begin{table}[!htbp]
\centering
\begin{tabular}{rrrr}
\toprule
Decimal places & N & Pct & Cum. pct \\
\midrule
8  & 60       & 0.00 & 0.00 \\
9  & 1{,}195  & 0.01 & 0.01 \\
10 & 18{,}766 & 0.12 & 0.13 \\
11 & 139{,}816 & 0.89 & 1.02 \\
12 & 1{,}539{,}582 & 9.80 & 10.82 \\
13 & 14{,}013{,}913 & 89.18 & 100.00 \\
15 & 1        & 0.00 & 100.00 \\
\bottomrule
\end{tabular}
\caption{Latitude precision distribution (decimal places). Low precision threshold: $<10$ decimals. Low precision total: 1{,}255 (0.01\%).}
\label{tab:latitude-precision-summary}
\end{table}

% =========================================================
\subsection{Low-precision latitude by agency (top 10 as printed)}

\begin{table}[!htbp]
\centering
\begin{tabular}{lrrr}
\toprule
Agency & N & Pct & Cum. pct \\
\midrule
NYPD  & 536 & 42.71 & 42.71 \\
DSNY  & 199 & 15.86 & 58.57 \\
HPD   & 127 & 10.12 & 68.69 \\
DOB   & 96  & 7.65  & 76.34 \\
DOT   & 72  & 5.74  & 82.08 \\
DEP   & 63  & 5.02  & 87.10 \\
DOHMH & 58  & 4.62  & 91.72 \\
DPR   & 56  & 4.46  & 96.18 \\
OSE   & 13  & 1.04  & 97.22 \\
DHS   & 11  & 0.88  & 98.10 \\
\bottomrule
\end{tabular}
\caption{Low-precision latitude records ($<10$ decimals) by agency (top 10 as printed).}
\label{tab:low-precision-latitude-by-agency-top10}
\end{table}

% =========================================================
\subsection{Longitude precision summary (decimal places)}

\begin{table}[!htbp]
\centering
\begin{tabular}{rrrr}
\toprule
Decimal places & N & Pct & Cum. pct \\
\midrule
0  & 216     & 0.00 & 0.00 \\
7  & 7       & 0.00 & 0.00 \\
8  & 14      & 0.00 & 0.00 \\
9  & 1{,}219 & 0.01 & 0.01 \\
10 & 12{,}307 & 0.08 & 0.09 \\
11 & 138{,}188 & 0.88 & 0.97 \\
12 & 1{,}382{,}778 & 8.80 & 9.77 \\
13 & 14{,}178{,}603 & 90.23 & 100.00 \\
15 & 1       & 0.00 & 100.00 \\
\bottomrule
\end{tabular}
\caption{Longitude precision distribution (decimal places). Low precision threshold: $<10$ decimals. Low precision total: 1{,}456 (0.01\%).}
\label{tab:longitude-precision-summary}
\end{table}

% =========================================================
\subsection{Low-precision longitude by agency (top 10 as printed)}

\begin{table}[!htbp]
\centering
\begin{tabular}{lrrr}
\toprule
Agency & N & Pct & Cum. pct \\
\midrule
NYPD  & 648 & 44.51 & 44.51 \\
DSNY  & 217 & 14.90 & 59.41 \\
HPD   & 213 & 14.63 & 74.04 \\
DEP   & 107 & 7.35  & 81.39 \\
DOB   & 65  & 4.46  & 85.85 \\
DPR   & 58  & 3.98  & 89.83 \\
DOT   & 50  & 3.43  & 93.26 \\
DOHMH & 45  & 3.09  & 96.35 \\
DCWP  & 18  & 1.24  & 97.59 \\
TLC   & 12  & 0.82  & 98.41 \\
\bottomrule
\end{tabular}
\caption{Low-precision longitude records ($<10$ decimals) by agency (top 10 as printed).}
\label{tab:low-precision-longitude-by-agency-top10}
\end{table}

% =========================================================
\subsection{Side-by-side precision comparison for latitude and longitude}

\begin{table}[!htbp]
\centering
\begin{tabular}{r r r r r r r}
\toprule
Dec. places & Lat N & Lat pct & Lon N & Lon pct & Lat cum. pct & Lon cum. pct \\
\midrule
0   & 0        & 0.00 & 216      & 0.00 & 0.00  & 0.00 \\
7   & 0        & 0.00 & 7        & 0.00 & 0.00  & 0.00 \\
8   & 60       & 0.00 & 14       & 0.00 & 0.00  & 0.00 \\
9   & 1{,}195  & 0.01 & 1{,}219  & 0.01 & 0.01  & 0.01 \\
10  & 18{,}766 & 0.12 & 12{,}307 & 0.08 & 0.13  & 0.09 \\
11  & 139{,}816 & 0.89 & 138{,}188 & 0.88 & 1.02  & 0.97 \\
12  & 1{,}539{,}582 & 9.80 & 1{,}382{,}778 & 8.80 & 10.82 & 9.77 \\
13  & 14{,}013{,}913 & 89.18 & 14{,}178{,}603 & 90.23 & 100.00 & 100.00 \\
15  & 1        & 0.00 & 1        & 0.00 & 100.00 & 100.00 \\
\midrule
TOTAL & 15{,}713{,}333 & 100.00 & 15{,}713{,}333 & 100.00 & NA & NA \\
\bottomrule
\end{tabular}
\caption{Side-by-side comparison of decimal-place precision for latitude and longitude.}
\label{tab:lat-lon-precision-side-by-side}
\end{table}

% =========================================================
\subsection{Duplicate/redundancy checks (summary)}

\begin{table}[!htbp]
\centering
\begin{tabular}{l l r r r}
\toprule
Reference & Duplicate & Rows & Duplication & Non-matches \\
\midrule
\texttt{borough} & \texttt{park\_borough} & 16{,}016{,}700 & 100.00\% & 0.00\% \\
\texttt{borough} & \texttt{taxi\_company\_borough} & 16{,}016{,}700 & 0.29\% & 99.71\% \\
\bottomrule
\end{tabular}
\caption{Duplicate/redundancy checks comparing candidate duplicate fields against reference fields.}
\label{tab:duplicate-redundancy-checks}
\end{table}

% =========================================================
\subsection{Status vs \texttt{closed\_date} inconsistency counts (exceptions only)}

\begin{table}[!htbp]
\centering
\begin{tabular}{l r r}
\toprule
Exception type & Count & Percent \\
\midrule
CLOSED status but missing \texttt{closed\_date} & 47{,}910 & 0.304\% of CLOSED \\
NOT CLOSED status but has \texttt{closed\_date} & 59{,}661 & 0.378\% of has-\texttt{closed\_date} \\
\bottomrule
\end{tabular}
\caption{Exceptions in consistency checks between \texttt{status} and \texttt{closed\_date}. Denominator A (CLOSED status): 15{,}774{,}141. Denominator B (has \texttt{closed\_date}): 15{,}785{,}892.}
\label{tab:status-vs-closed-date-exceptions}
\end{table}

% =========================================================
\subsection{CLOSED status but missing \texttt{closed\_date} by agency (as printed)}

\begin{table}[!htbp]
\centering
\begin{tabular}{lr}
\toprule
Agency & N \\
\midrule
DHS  & 47{,}755 \\
DOT  & 108 \\
DSNY & 47 \\
\bottomrule
\end{tabular}
\caption{CLOSED status but missing \texttt{closed\_date}, by agency (as printed).}
\label{tab:closed-status-missing-closed-date-by-agency}
\end{table}

% =========================================================
\subsection{NOT CLOSED status but has \texttt{closed\_date} by agency (top as printed)}

\begin{table}[!htbp]
\centering
\begin{tabular}{lr}
\toprule
Agency & N \\
\midrule
DOT   & 45{,}885 \\
DOB   & 9{,}257 \\
DSNY  & 3{,}702 \\
DEP   & 328 \\
NYPD  & 143 \\
DOHMH & 131 \\
DPR   & 105 \\
TLC   & 57 \\
HPD   & 27 \\
DCWP  & 21 \\
\bottomrule
\end{tabular}
\caption{NOT CLOSED status but has \texttt{closed\_date}, by agency (top as printed).}
\label{tab:not-closed-has-closed-date-by-agency-top}
\end{table}

% =========================================================
\subsection{Out-of-bounds coordinate checks (as printed)}

\begin{table}[!htbp]
\centering
\begin{tabular}{l r l l}
\toprule
Check & N & \texttt{unique\_key} & Agency \\
\midrule
Latitude out-of-bounds  & 1 & 52308247 & DHS \\
Longitude out-of-bounds & 1 & 52308247 & DHS \\
Both lat \& lon out-of-bounds & 1 & 52308247 & DHS \\
\bottomrule
\end{tabular}
\caption{Out-of-bounds checks for latitude/longitude (raw coords, $\pm$100 m buffer). The single offending record reported the same \texttt{unique\_key} across checks.}
\label{tab:lat-lon-out-of-bounds}
\end{table}

\begin{table}[!htbp]
\centering
\begin{tabular}{l r l l}
\toprule
Field & N & \texttt{unique\_key} & Agency \\
\midrule
\texttt{x\_coordinate\_state\_plane} out-of-bounds & 1 & 52308247 & DHS \\
\texttt{y\_coordinate\_state\_plane} out-of-bounds & 1 & 52308247 & DHS \\
\bottomrule
\end{tabular}
\caption{Out-of-bounds checks for state plane coordinates (as printed).}
\label{tab:state-plane-out-of-bounds}
\end{table}

% =========================================================
\subsection{Validation summary for allowable/valid values (sorted by invalid desc)}

\begin{table}[!htbp]
\centering
\begin{tabular}{l l r r r}
\toprule
Field & Valid & Invalid & Nonblank & Pct invalid \\
\midrule
\texttt{community\_board} & FALSE & 56{,}914 & 15{,}978{,}265 & 0.36 \\
\texttt{incident\_zip}    & FALSE & 3{,}274  & 15{,}757{,}103 & 0.02 \\
\texttt{address\_type}    & TRUE  & 0 & 13{,}246{,}948 & 0.00 \\
\texttt{agency}           & TRUE  & 0 & 16{,}016{,}700 & 0.00 \\
\texttt{borough}          & TRUE  & 0 & 15{,}978{,}265 & 0.00 \\
\texttt{open\_data\_channel\_type} & TRUE & 0 & 16{,}016{,}700 & 0.00 \\
\texttt{park\_borough}    & TRUE  & 0 & 15{,}978{,}265 & 0.00 \\
\texttt{status}           & TRUE  & 0 & 16{,}016{,}700 & 0.00 \\
\texttt{taxi\_company\_borough} & TRUE & 0 & 8{,}564 & 0.00 \\
\texttt{vehicle\_type}    & TRUE  & 0 & 181{,}533 & 0.00 \\
\bottomrule
\end{tabular}
\caption{Validation summary for selected categorical fields (as printed).}
\label{tab:validation-summary-fields}
\end{table}
%=========================================

%
% =========================================================
\subsection{Date Field Analysis}

\begin{table}[!htbp]
\centering
\begin{tabular}{lrrr}
\toprule
Year & Count & Percentage (\%) & Cumulative (\%) \\
\midrule
2019 & 514 & 0.00 & 0.00 \\
2020 & 2{,}793{,}192 & 17.55 & 17.56 \\
2021 & 3{,}138{,}306 & 19.72 & 37.28 \\
2022 & 3{,}061{,}230 & 19.24 & 56.52 \\
2023 & 3{,}183{,}762 & 20.01 & 76.53 \\
2024 & 3{,}542{,}369 & 22.26 & 98.79 \\
2025 & 192{,}242 & 1.21 & 100.00 \\
\bottomrule
\end{tabular}
\caption{Year distribution for \texttt{resolution\_action\_updated\_date}.}
\label{tab:year-dist-resolution-action}
\end{table}

\begin{table}[!htbp]
\centering
\begin{tabular}{lrrr}
\toprule
Year & Count & Percentage (\%) & Cumulative (\%) \\
\midrule
2018 & 3 & 0.01 & 0.01 \\
2019 & 13 & 0.02 & 0.03 \\
2020 & 833 & 1.59 & 1.62 \\
2021 & 3{,}218 & 6.15 & 7.77 \\
2022 & 13{,}189 & 25.20 & 32.97 \\
2023 & 13{,}910 & 26.57 & 59.54 \\
2024 & 19{,}640 & 37.52 & 97.06 \\
2025 & 1{,}540 & 2.94 & 100.00 \\
\bottomrule
\end{tabular}
\caption{Year distribution for \texttt{due\_date}.}
\label{tab:year-dist-due-date}
\end{table}



\begin{table}[!htbp]
\centering
\begin{tabular}{lrrr}
\toprule
Year & Count & Percentage (\%) & Cumulative (\%) \\
\midrule
2020 & 2{,}942{,}064 & 18.37 & 18.37 \\
2021 & 3{,}220{,}915 & 20.11 & 38.48 \\
2022 & 3{,}169{,}844 & 19.79 & 58.27 \\
2023 & 3{,}225{,}557 & 20.14 & 78.41 \\
2024 & 3{,}458{,}320 & 21.59 & 100.00 \\
\bottomrule
\end{tabular}
\caption{Year distribution for \texttt{created\_date}.}
\label{tab:year-dist-created-date}
\end{table}


\begin{table}[!htbp]
\centering
\begin{tabular}{lrrr}
\toprule
Year & Count & Percentage (\%) & Cumulative (\%) \\
\midrule
1899 & 103 & 0.00 & 0.00 \\
1900 & 11 & 0.00 & 0.00 \\
2004 & 1 & 0.00 & 0.00 \\
2019 & 523 & 0.00 & 0.00 \\
2020 & 2{,}797{,}498 & 17.72 & 17.73 \\
2021 & 3{,}118{,}070 & 19.75 & 37.48 \\
2022 & 3{,}047{,}652 & 19.31 & 56.78 \\
2023 & 3{,}143{,}900 & 19.92 & 76.70 \\
2024 & 3{,}483{,}764 & 22.07 & 98.77 \\
2025 & 194{,}369 & 1.23 & 100.00 \\
2033 & 1 & 0.00 & 100.00 \\
\bottomrule
\end{tabular}
\caption{Year distribution for \texttt{closed\_date}.}
\label{tab:year-dist-closed-date}
\end{table}


\begin{table}[!htbp]
\centering
\begin{tabular}{lrr}
\toprule
Anomaly Type & Count & Percentage (\%) \\
\midrule
Future created dates & 0 & 0.00 \\
Past created dates & 0 & 0.00 \\
Midnight-only created & 14{,}409 & 0.09 \\
Noon-only created & 31{,}872 & 0.20 \\
\bottomrule
\end{tabular}
\caption{Temporal anomalies in \texttt{created\_date} field.}
\label{tab:created-date-anomalies}
\end{table}



\begin{table}[!htbp]
\centering
\begin{tabular}{lrrr}
\toprule
Agency & Count & Percentage (\%) & Cumulative (\%) \\
\midrule
DOHMH & 11{,}121 & 77.18 & 77.18 \\
DOT & 2{,}728 & 18.93 & 96.11 \\
DSNY & 216 & 1.50 & 97.61 \\
DEP & 194 & 1.35 & 98.96 \\
NYPD & 132 & 0.92 & 99.87 \\
HPD & 10 & 0.07 & 99.94 \\
DOB & 4 & 0.03 & 99.97 \\
OSE & 2 & 0.01 & 99.99 \\
EDC & 1 & 0.01 & 99.99 \\
DHS & 1 & 0.01 & 100.00 \\
\bottomrule
\end{tabular}
\caption{Agency distribution for midnight \texttt{created\_date} entries (N = 14{,}409).}
\label{tab:midnight-created-by-agency}
\end{table}



\begin{table}[!htbp]
\centering
\begin{tabular}{lrrr}
\toprule
Agency & Count & Percentage (\%) & Cumulative (\%) \\
\midrule
DSNY & 30{,}733 & 96.43 & 96.43 \\
DEP & 505 & 1.58 & 98.01 \\
DOT & 487 & 1.53 & 99.54 \\
NYPD & 55 & 0.17 & 99.71 \\
HPD & 54 & 0.17 & 99.88 \\
DPR & 11 & 0.03 & 99.92 \\
DOHMH & 10 & 0.03 & 99.95 \\
DOB & 5 & 0.02 & 99.96 \\
DHS & 4 & 0.01 & 99.98 \\
TLC & 3 & 0.01 & 99.99 \\
DCWP & 2 & 0.01 & 99.99 \\
OSE & 2 & 0.01 & 100.00 \\
EDC & 1 & 0.00 & 100.00 \\
\bottomrule
\end{tabular}
\caption{Agency distribution for noon \texttt{created\_date} entries (N = 31{,}872).}
\label{tab:noon-created-by-agency}
\end{table}


\begin{table}[!htbp]
\centering
\begin{tabular}{lrr}
\toprule
Anomaly Type & Count & Percentage (\%) \\
\midrule
Future dates & 1 & 0.00 \\
Past dates & 0 & 0.00 \\
Missing dates & 15{,}964{,}354 & 99.67 \\
Midnight-only & 0 & 0.00 \\
Noon-only & 1 & 0.00 \\
Due $<$ Created & 16 & 0.00 \\
\bottomrule
\end{tabular}
\caption{Temporal anomalies in \texttt{due\_date} field.}
\label{tab:due-date-anomalies}
\end{table}



\begin{table}[!htbp]
\centering
\begin{tabular}{lr}
\toprule
Agency & Missing Due Dates \\
\midrule
NYPD & 6{,}905{,}945 \\
HPD & 3{,}067{,}970 \\
DSNY & 1{,}871{,}159 \\
DOT & 1{,}078{,}415 \\
DEP & 834{,}539 \\
DPR & 636{,}283 \\
DOB & 470{,}508 \\
DOHMH & 397{,}968 \\
DHS & 211{,}486 \\
EDC & 150{,}305 \\
\bottomrule
\end{tabular}
\caption{Top agencies with missing \texttt{due\_date} values (total missing: 15{,}964{,}354).}
\label{tab:missing-due-dates-by-agency}
\end{table}



\paragraph{Interpretation.}
The temporal analysis of NYC 311 service request reveals several data quality 
concerns across date fields. The \texttt{created\_date} field shows consistent coverage 
from 2020--2024 with roughly 3 million records annually, though 14,409 
records (0.09\%) have midnight-only timestamps and 31,872 (0.20\%) have noon-only 
timestamps. These truncated timestamps suggest batch processing or data entry 
artifacts, with DOHMH accounting for 77\% of midnight entries and DSNY representing 96\% 
of noon entries. The \texttt{closed\_date} field exhibits more severe anomalies, 
including 103 records from 1899, 11 from 1900, and one from 2033, indicating clear 
data entry errors or system failures. The \texttt{due\_date} field is nearly absent from 
the dataset, with 99.67\% missing values concentrated in NYPD (6.9 million), 
HPD (3.1 million), and DSNY (1.9 million) records. This extreme sparsity suggests that 
due dates are either not assigned for most service request types or are tracked through 
separate systems. The 16 records where \texttt{due\_date} $<$ \texttt{created\_date} 
represent logical impossibilities that warrant investigation.


\begin{table}[!htbp]
\centering
\begin{tabular}{lr}
\toprule
Metric & Value \\
\midrule
Non-blank \texttt{due\_date} rows & 52{,}346 \\
Due dates before created date & 16 \\
Percentage of non-blank & 0.03\% \\
\midrule
\multicolumn{2}{l}{\textit{Duration statistics (days):}} \\
Minimum & $-727.26$ \\
Maximum & $-451.70$ \\
Mean & $-557.97$ \\
Median & $-549.07$ \\
Standard deviation & 76.86 \\
\bottomrule
\end{tabular}
\caption{Summary of service requests with \texttt{due\_date} before \texttt{created\_date}.}
\label{tab:due-before-created-summary}
\end{table}



\begin{table}[!htbp]
\centering
\begin{tabular}{lllr}
\toprule
Agency & Created Date & Due Date & Duration (days) \\
\midrule
DPR & 2020-09-25 15:45:06 & 2019-03-12 13:39:06 & $-563.09$ \\
DPR & 2020-09-24 13:24:00 & 2019-02-06 10:00:48 & $-596.10$ \\
DPR & 2020-09-25 15:43:34 & 2018-09-29 09:26:39 & $-727.26$ \\
DPR & 2020-09-25 16:00:27 & 2019-02-26 17:46:25 & $-576.88$ \\
DPR & 2020-09-25 15:59:20 & 2019-04-09 10:26:17 & $-535.23$ \\
DPR & 2020-09-25 15:10:23 & 2019-03-14 11:38:19 & $-561.15$ \\
DPR & 2020-09-25 15:56:36 & 2019-06-26 13:51:24 & $-457.09$ \\
DPR & 2020-09-25 16:01:34 & 2019-04-13 15:23:00 & $-531.03$ \\
DPR & 2020-10-23 15:36:37 & 2019-04-24 16:28:16 & $-547.96$ \\
DPR & 2020-09-25 16:02:56 & 2019-04-13 15:11:15 & $-531.04$ \\
\bottomrule
\end{tabular}
\caption{Sample records with \texttt{due\_date} before \texttt{created\_date} (N = 16, all from DPR).}
\label{tab:due-before-created-examples}
\end{table}


\paragraph{Interpretation}
Among the 52,346 service requests with non-blank \texttt{due\_date} values, 16 
records (0.03\%) exhibit the logical impossibility of having due dates that precede 
their created dates. All 16 anomalous records originated from the Department of 
Parks and Recreation (DPR), with negative durations ranging from $-727.26$ to $-451.70$ 
days (mean = $-557.97$ days, SD = 76.86 days). The temporal pattern is striking: 15 of 
the 16 records share created dates clustered on September 25, 2020, while their due 
dates span from September 2018 to June 2019. This systematic error suggests a batch 
data processing issue or database migration problem specific to DPR records during 
late September 2020, where due dates from previous years were incorrectly associated 
with newly created service request entries. The approximately 18-month gap between 
the due dates and created dates indicates these may represent historical records 
that were retroactively entered into the system with corrupted temporal metadata.


\subsection{Resolution action updated date anomaly summary}

The following table summarizes validation checks applied to the
\texttt{resolution\_action\_updated\_date} field, using
\texttt{America/New\_York} as the evaluation time zone.

\begin{table}[!htbp]
\centering
\begin{tabular}{lrr}
\toprule
Anomaly type & Count (N) & Percent (\%) \\
\midrule
Future dates                & 0        & 0.00 \\
Past dates                  & 0        & 0.00 \\
Missing dates               & 105{,}085 & 0.66 \\
Midnight-only timestamps    & 3{,}773{,}498 & 23.56 \\
Noon-only timestamps        & 475{,}906 & 2.97 \\
Updated $<$ Created         & 374{,}974 & 2.34 \\
\bottomrule
\end{tabular}
\caption{Anomaly summary for \texttt{resolution\_action\_updated\_date}
(analysis time zone: \texttt{America/New\_York}).}
\label{tab:resolution-update-date-anomalies}
\end{table}


\subsection{Midnight \texttt{resolution\_action\_updated\_date} values by agency}

The following table summarizes service requests whose
\texttt{resolution\_action\_updated\_date} timestamp occurs exactly at midnight
(\texttt{00:00:00}), stratified by reporting agency. A Pareto chart based on
these results is provided separately.

\begin{table}[!htbp]
\centering
\begin{tabular}{lrrr}
\toprule
Agency & Count (N) & Percent (\%) & Cumulative (\%) \\
\midrule
HPD   & 2{,}661{,}746 & 70.53 & 70.53 \\
DSNY  &   626{,}551 & 16.61 & 87.14 \\
DOB   &   470{,}455 & 12.46 & 99.60 \\
DOT   &     8{,}159 & 0.22  & 99.82 \\
DEP   &     5{,}920 & 0.16  & 99.98 \\
DOHMH &       586 & 0.02  & 100.00 \\
NYPD  &        76 & 0.00  & 100.00 \\
DCWP  &         5 & 0.00  & 100.00 \\
\bottomrule
\end{tabular}
\caption{Pareto summary of midnight \texttt{resolution\_action\_updated\_date}
timestamps by agency (total observations: 3{,}773{,}498; agencies represented: 8).}
\label{tab:midnight-resolution-action-updated-pareto}
\end{table}


\subsection{Noon \texttt{resolution\_action\_updated\_date} values by agency}

This table summarizes service requests whose
\texttt{resolution\_action\_updated\_date} timestamp occurs exactly at noon
(\texttt{12:00:00}), stratified by reporting agency. Results are sorted in
descending order by volume to support Pareto-style interpretation.

\begin{table}[!htbp]
\centering
\begin{tabular}{lrrr}
\toprule
Agency & Count (N) & Percent (\%) & Cumulative (\%) \\
\midrule
DSNY  &   461{,}017 & 96.88 & 96.88 \\
DEP   &    11{,}268 & 2.37  & 99.25 \\
DOT   &     3{,}323 & 0.70  & 99.95 \\
DOHMH &       175 & 0.04  & 99.99 \\
NYPD  &        74 & 0.02  & 100.00 \\
OSE   &        24 & 0.01  & 100.00 \\
DPR   &        11 & 0.00  & 100.00 \\
EDC   &         7 & 0.00  & 100.00 \\
HPD   &         3 & 0.00  & 100.00 \\
TLC   &         2 & 0.00  & 100.00 \\
DHS   &         1 & 0.00  & 100.00 \\
DOE   &         1 & 0.00  & 100.00 \\
\bottomrule
\end{tabular}
\caption{Pareto summary of noon \texttt{resolution\_action\_updated\_date}
timestamps by agency (total observations: 475{,}906; agencies represented: 12).}
\label{tab:noon-resolution-action-updated-pareto}
\end{table}

\subsection{\texttt{resolution\_action\_updated\_date} earlier than \texttt{created\_date}}

This table summarizes records for which the
\texttt{resolution\_action\_updated\_date} precedes the
\texttt{created\_date}, representing a logical inconsistency.
Results are stratified by reporting agency and sorted in descending
order by volume to support Pareto-style interpretation.

\begin{table}[!htbp]
\centering
\begin{tabular}{lrrr}
\toprule
Agency & Count (N) & Percent (\%) & Cumulative (\%) \\
\midrule
HPD   & 160{,}740 & 42.87 & 42.87 \\
DOT   & 128{,}348 & 34.23 & 77.10 \\
DOB   &  61{,}080 & 16.29 & 93.39 \\
DSNY  &  23{,}371 &  6.23 & 99.62 \\
DEP   &     616 &  0.16 & 99.78 \\
NYPD  &     514 &  0.14 & 99.92 \\
DPR   &     200 &  0.05 & 99.97 \\
DOHMH &      66 &  0.02 & 99.99 \\
OSE   &      37 &  0.01 & 100.00 \\
DHS   &       2 &  0.00 & 100.00 \\
\bottomrule
\end{tabular}
\caption{Pareto summary of records where
\texttt{resolution\_action\_updated\_date} occurs earlier than
\texttt{created\_date} (total observations: 374{,}974; agencies represented: 10).}
\label{tab:resolution-updated-before-created-pareto}
\end{table}


\subsection{Missing \texttt{resolution\_action\_updated\_date} values by agency}

The following table reports agencies with missing
\texttt{resolution\_action\_updated\_date} values, sorted in descending
order by frequency. Only agencies appearing in the program output are
shown.

\begin{table}[!htbp]
\centering
\begin{tabular}{lr}
\toprule
Agency & Count (N) \\
\midrule
NYPD  & 27{,}261 \\
DOHMH & 22{,}694 \\
EDC   & 20{,}094 \\
DPR   & 17{,}021 \\
DSNY  &  5{,}892 \\
TLC   &  4{,}127 \\
HPD   &  2{,}949 \\
DEP   &  2{,}748 \\
DHS   &  1{,}038 \\
OSE   &    967 \\
\bottomrule
\end{tabular}
\caption{Top agencies with missing \texttt{resolution\_action\_updated\_date}
values (total missing records: 105{,}085).}
\label{tab:missing-resolution-action-updated-by-agency}
\end{table}


\subsection{\texttt{resolution\_action\_updated\_date} preceding \texttt{created\_date}: distribution}

This table summarizes the magnitude of temporal inconsistencies in which
\texttt{resolution\_action\_updated\_date} occurs before
\texttt{created\_date}. Negative values indicate the number of days by
which the update timestamp precedes record creation.

\begin{table}[!htbp]
\centering
\begin{tabular}{lr}
\toprule
Statistic & Value \\
\midrule
Total affected records & 374{,}974 \\
Percent of all records & 2.34\% \\
Minimum (days) & $-1{,}619.87$ \\
Maximum (days) & $-0.00$ \\
Mean (days)    & $-1.83$ \\
Median (days)  & $-0.49$ \\
Standard deviation (days) & 19.79 \\
\bottomrule
\end{tabular}
\caption{Summary statistics for records where
\texttt{resolution\_action\_updated\_date} precedes \texttt{created\_date}.
Negative values represent the number of days by which the update timestamp
occurs earlier than record creation.}
\label{tab:resolution-updated-before-created-stats}
\end{table}



\subsection{Example records with update timestamps preceding creation}

The following table presents a small illustrative sample of records
exhibiting negative update durations. The identifier field has been
omitted; examples are intended to demonstrate common patterns rather
than serve as exhaustive evidence.

\begin{table}[!htbp]
\centering
\begin{tabular}{l l l r}
\toprule
Agency & Created date & Resolution action updated date & Duration (days) \\
\midrule
HPD  & 2024-01-23 14:30:04 & 2024-01-23 00:00:00 & $-0.60$ \\
DOT  & 2023-01-12 11:51:00 & 2023-01-11 11:51:00 & $-1.00$ \\
DOB  & 2020-08-26 03:41:31 & 2020-08-26 00:00:00 & $-0.15$ \\
DOT  & 2020-07-02 15:13:00 & 2020-05-19 15:13:00 & $-44.00$ \\
NYPD & 2022-11-06 01:44:09 & 2022-11-06 01:06:43 & $-0.07$ \\
HPD  & 2022-02-28 09:32:11 & 2022-02-28 00:00:00 & $-0.40$ \\
DOT  & 2022-10-12 10:09:00 & 2022-10-11 10:08:00 & $-1.00$ \\
DSNY & 2021-07-01 13:51:00 & 2021-07-01 12:00:00 & $-0.08$ \\
DOB  & 2020-04-29 09:43:43 & 2020-04-29 00:00:00 & $-0.41$ \\
DOT  & 2022-12-20 08:02:00 & 2022-12-19 08:02:00 & $-1.00$ \\
\bottomrule
\end{tabular}
\caption{Illustrative examples of service requests where
\texttt{resolution\_action\_updated\_date} precedes \texttt{created\_date}.
Negative durations indicate the magnitude of the temporal inconsistency.}
\label{tab:resolution-updated-before-created-examples}
\end{table}
	



\subsection{Distribution of negative resolution update durations by agency}

This table reports summary statistics for negative values of
\texttt{resolution\_action\_updated\_date} minus \texttt{created\_date},
expressed in days, stratified by agency. Only agencies with at least one
negative-duration record are shown.

\begin{table}[!htbp]
\centering
\begin{tabular}{lrrrrrrrr}
\toprule
Agency & N & Min & Q1 & Median & Q3 & Max & Mean & SD \\
\midrule
DEP   &   199 & -472.71 & -154.42 & -0.30 & -0.03 & 0.00 & -84.88 & 145.75 \\
DOB   & 20{,}269 &  -1.00 &  -0.54 & -0.45 & -0.39 & 0.00 &  -0.46 &   0.15 \\
DOHMH &    12 & -793.20 & -420.59 & -194.80 & -0.41 & -0.01 & -239.04 & 270.67 \\
DOT   & 43{,}093 & -370.00 &  -1.34 & -1.00 & -1.00 & 0.00 &  -2.47 &  10.77 \\
DPR   &    64 &  -39.86 &   0.00 &  0.00 &  0.00 & 0.00 &  -1.06 &   5.97 \\
DSNY  &  7{,}888 & -856.56 &  -0.55 & -0.14 & -0.07 & 0.00 &  -7.62 &  46.15 \\
HPD   & 53{,}261 &  -1.04 &  -0.50 & -0.39 & -0.28 & 0.00 &  -0.39 &   0.19 \\
NYPD  &    181 &  -0.16 &  -0.04 & -0.02 &  0.00 & 0.00 &  -0.03 &   0.03 \\
OSE   &     15 &   0.00 &   0.00 &  0.00 &  0.00 & 0.00 &   0.00 &   0.00 \\
\bottomrule
\end{tabular}
\caption{Summary statistics for negative resolution update durations
(\texttt{resolution\_action\_updated\_date} earlier than
\texttt{created\_date}), by agency. Durations are expressed in days;
negative values indicate the magnitude of the temporal inconsistency.}
\label{tab:negative-resolution-update-distribution-by-agency}
\end{table}



\subsection{Pareto distribution of negative resolution update records by agency}

The following table summarizes the volume of records with negative
resolution update durations, stratified by agency and sorted in
descending order to support Pareto-style interpretation.

\begin{table}[!htbp]
\centering
\begin{tabular}{lrrr}
\toprule
Agency & Count (N) & Percent (\%) & Cumulative (\%) \\
\midrule
HPD   & 160{,}740 & 42.87 & 42.87 \\
DOT   & 128{,}348 & 34.23 & 77.10 \\
DOB   &  61{,}080 & 16.29 & 93.39 \\
DSNY  &  23{,}371 &  6.23 & 99.62 \\
DEP   &     616 &  0.16 & 99.78 \\
NYPD  &     514 &  0.14 & 99.92 \\
DPR   &     200 &  0.05 & 99.97 \\
DOHMH &      66 &  0.02 & 99.99 \\
OSE   &      37 &  0.01 & 100.00 \\
DHS   &       2 &  0.00 & 100.00 \\
\bottomrule
\end{tabular}
\caption{Pareto summary of records where
\texttt{resolution\_action\_updated\_date} precedes
\texttt{created\_date}, by agency (total observations: 374{,}974).}
\label{tab:negative-resolution-update-pareto-by-agency}
\end{table}


\paragraph{Synthesis and interpretation.}
Taken together, Tables~\ref{tab:negative-resolution-update-distribution-by-agency}
and~\ref{tab:negative-resolution-update-pareto-by-agency} reveal two distinct
patterns underlying negative resolution update durations. Agencies such as
HPD, DOT, and DOB account for the majority of affected records by volume, yet
their median and interquartile ranges are tightly concentrated near zero,
indicating systematic but small temporal offsets consistent with truncated
timestamps or batch update processes. In contrast, agencies with relatively few
violations (notably DEP, DOHMH, and DSNY) exhibit extreme negative durations,
with minima spanning hundreds of days, suggesting sporadic data-entry errors
or system-level failures rather than routine processing artifacts. This
contrast underscores that both frequency and severity must be considered when
assessing data quality: high-volume, low-magnitude anomalies point to workflow
conventions, whereas low-frequency, high-magnitude anomalies signal exceptional
events requiring targeted investigation. Together, these findings reinforce
the need for agency-specific validation and remediation strategies rather than
uniform, one-size-fits-all rules.


\subsection{Post-closed resolution updates: overall distribution}

This table summarizes all cases in which
\texttt{resolution\_action\_updated\_date} occurs after
\texttt{closed\_date}, regardless of magnitude. Durations are expressed
in days.

\begin{table}[!htbp]
\centering
\begin{tabular}{lr}
\toprule
Statistic & Value \\
\midrule
Total post-closed updates & 8{,}576{,}533 \\
Percent of all records   & 53.55\% \\
Minimum (days)           & 0.00001 \\
First quartile (days)    & 0.00003 \\
Median (days)            & 0.00005 \\
Mean (days)              & 0.96 \\
Third quartile (days)    & 0.00006 \\
Maximum (days)           & 45{,}216.61 \\
\bottomrule
\end{tabular}
\caption{Distribution of all post-closed resolution update durations.
Most updates occur within seconds of closure, while a small fraction
exhibit very large delays.}
\label{tab:post-closed-updates-overall}
\end{table}


\subsection{Late post-closed resolution updates}

This table summarizes records for which
\texttt{resolution\_action\_updated\_date} occurs at least 30 days after
\texttt{closed\_date}, representing substantively delayed updates.

\begin{table}[!htbp]
\centering
\begin{tabular}{lr}
\toprule
Statistic & Value \\
\midrule
Total late updates        & 27{,}616 \\
Percent of all records    & 0.17\% \\
Minimum (days)            & 30.00 \\
First quartile (days)     & 48.79 \\
Median (days)             & 93.45 \\
Mean (days)               & 127.58 \\
Third quartile (days)     & 162.29 \\
Maximum (days)            & 1{,}094.09 \\
Negative values observed  & 0 \\
\bottomrule
\end{tabular}
\caption{Distribution of late post-closed resolution update durations
(\(\geq 30\) days). All values are positive, indicating updates occurring
well after record closure.}
\label{tab:post-closed-updates-late}
\end{table}


\subsection{Extremely late post-closed resolution updates}

This table reports summary statistics for a small subset of records
exhibiting extremely delayed post-closure updates, measured in years
rather than days.

\begin{table}[!htbp]
\centering
\begin{tabular}{lr}
\toprule
Statistic & Value \\
\midrule
Total extremely late updates & 248 \\
Percent of all records       & $<0.01$\% \\
Minimum (days)               & 1{,}096.40 \\
First quartile (days)        & 1{,}153.31 \\
Median (days)                & 1{,}404.82 \\
Mean (days)                  & 16{,}960.65 \\
Third quartile (days)        & 44{,}376.93 \\
Maximum (days)               & 45{,}216.61 \\
\bottomrule
\end{tabular}
\caption{Summary statistics for extremely late post-closed resolution
updates, representing multi-year delays between closure and subsequent
record modification.}
\label{tab:post-closed-updates-extreme}
\end{table}


\paragraph{Synthesis and interpretation.}
Post-closed resolution updates are common in the NYC~311 dataset, with
over half of all records exhibiting some degree of post-closure
modification (Table~\ref{tab:post-closed-updates-overall}). However, the
vast majority of these updates occur within seconds of closure and are
consistent with routine system writes, metadata synchronization, or
batch-processing artifacts rather than substantive content changes.
When attention is restricted to materially delayed updates
(Table~\ref{tab:post-closed-updates-late}), the affected population
shrinks dramatically to less than one percent of all records, revealing
delays on the order of weeks to years that may reflect reopening,
auditing, or corrective workflows. A very small subset of records
(Table~\ref{tab:post-closed-updates-extreme}) exhibits multi-year delays,
suggesting exceptional events such as system migrations, retroactive
data repairs, or legacy record reconciliation. Together, these findings
demonstrate that post-closure updates are not inherently anomalous, but
their magnitude provides a critical signal for distinguishing benign
processing behavior from substantive data-quality concerns.



\subsection{Distribution of late post-closed update durations (histogram summary)}

This subsection summarizes the distribution of late post-closed update
durations (\texttt{postClosedUpdateDuration} $> 30$ days), corresponding
to the population visualized in the associated histogram. Summary
statistics are reported for the untrimmed distribution, with a small
fraction of extreme values excluded from the histogram display for
visual clarity.

\begin{table}[!htbp]
\centering
\begin{tabular}{lr}
\toprule
Statistic & Value \\
\midrule
Total records & 27{,}616 \\
Minimum (days) & 30.00 \\
Maximum (days) & 1{,}094.09 \\
Median (days)  & 93.45 \\
Mean (days)    & 127.58 \\
Standard deviation (days) & 124.51 \\
Trimmed observations & 277 (1.0\%) \\
\bottomrule
\end{tabular}
\caption{Summary statistics for late post-closed resolution update
durations (\(>30\) and \(<1095\) days), corresponding to the histogram of
\texttt{postClosedUpdateDuration}.}
\label{tab:post-closed-updates-late-histogram-summary}
\end{table}


\paragraph{Contextual breakdown.}
Across the full dataset, post-closed resolution updates occur frequently,
with 8{,}576{,}533 records exhibiting positive update durations. However,
only a small fraction of these represent substantively delayed updates:
27{,}616 records fall into the late category (\(>30\) and \(\leq 1095\)
days), and just 248 records meet the threshold for extreme delays
(\(\geq 1095\) days). The histogram therefore isolates the middle tier of
interest, where update delays are operationally meaningful but not
dominated by multi-year outliers.


\subsection{Overall magnitude of positive post-closed update durations}

This table reports high-level summary statistics for all records with
positive post-closed update durations
(\texttt{resolution\_action\_updated\_date} later than
\texttt{closed\_date}), expressed in days.

\begin{table}[!htbp]
\centering
\begin{tabular}{lr}
\toprule
Statistic & Value \\
\midrule
Median (days) & 0.00 \\
Mean (days)   & 0.96 \\
Maximum (days) & 45{,}216.61 \\
\bottomrule
\end{tabular}
\caption{Overall summary statistics for positive post-closed resolution
update durations. While the median delay is effectively zero, the mean
and maximum are influenced by a small number of extremely delayed
updates.}
\label{tab:post-closed-updates-overall-stats}
\end{table}


\subsection{Late post-closed resolution updates by agency}

This table summarizes agencies associated with late post-closed resolution
updates, defined as cases where
\texttt{resolution\_action\_updated\_date} occurs more than 30 days and no
more than 1095 days after \texttt{closed\_date}. Agencies are ordered by
descending frequency to support Pareto-style interpretation.

\begin{table}[!htbp]
\centering
\begin{tabular}{lrrr}
\toprule
Agency & Count (N) & Percent (\%) & Cumulative (\%) \\
\midrule
TLC   &  9{,}948 & 36.03 & 36.03 \\
DPR   &  8{,}027 & 29.07 & 65.10 \\
DOHMH &  5{,}463 & 19.79 & 84.89 \\
DSNY  &  3{,}020 & 10.93 & 95.82 \\
DOB   &    414 &  1.50 & 97.32 \\
DEP   &    346 &  1.25 & 98.57 \\
DOT   &    237 &  0.86 & 99.43 \\
DCWP  &    159 &  0.58 & 100.00 \\
DOE   &      2 &  0.01 & 100.00 \\
\bottomrule
\end{tabular}
\caption{Pareto distribution of late post-closed resolution updates
(30–1095 days after closure), by agency (total records: 27{,}616).}
\label{tab:late-postclosed-updates-pareto}
\end{table}


\subsection{Example late post-closed resolution updates}

The following table presents a small illustrative sample of late
post-closed updates. Examples are shown to highlight typical magnitudes
and patterns and do not represent an exhaustive listing.

\begin{table}[!htbp]
\centering
\begin{tabular}{l l l r}
\toprule
Agency & Closed date & Resolution action updated date & Delay (days) \\
\midrule
DSNY  & 2022-03-10 00:00:00 & 2022-04-18 15:40:51 &  39.61 \\
DPR   & 2021-07-14 08:39:24 & 2021-11-01 17:35:53 & 110.37 \\
DOHMH & 2024-06-07 19:46:29 & 2024-07-10 10:02:32 &  32.59 \\
DSNY  & 2024-03-22 00:00:00 & 2024-12-14 10:37:27 & 267.48 \\
DPR   & 2020-12-15 11:05:54 & 2021-11-01 16:54:05 & 321.20 \\
DPR   & 2021-09-02 09:08:45 & 2021-11-01 15:08:08 &  60.25 \\
DPR   & 2021-06-28 14:02:25 & 2021-11-02 14:59:15 & 127.04 \\
TLC   & 2022-07-15 11:18:01 & 2022-11-21 14:39:25 & 129.18 \\
DSNY  & 2024-04-15 00:00:00 & 2024-12-14 10:37:31 & 243.48 \\
DPR   & 2021-07-09 14:07:29 & 2021-11-02 15:35:32 & 116.06 \\
\bottomrule
\end{tabular}
\caption{Illustrative examples of late post-closed resolution updates
(30–1095 days after closure).}
\label{tab:late-postclosed-updates-examples}
\end{table}


\subsection{Extremely late post-closed resolution updates by agency}

This table summarizes agencies associated with extremely late post-closed
updates, defined as cases where
\texttt{resolution\_action\_updated\_date} occurs at least 1095 days after
\texttt{closed\_date}. Only agencies with five or more such records are
retained for plotting and interpretation.

\begin{table}[!htbp]
\centering
\begin{tabular}{lrrr}
\toprule
Agency & Count (N) & Percent (\%) & Cumulative (\%) \\
\midrule
DOT   & 104 & 41.94 & 41.94 \\
TLC   &  88 & 35.48 & 77.42 \\
DPR   &  37 & 14.92 & 92.34 \\
DHS   &   8 &  3.23 & 95.57 \\
DEP   &   6 &  2.42 & 97.99 \\
DSNY  &   2 &  0.81 & 98.80 \\
DOHMH &   2 &  0.81 & 99.61 \\
DCWP  &   1 &  0.40 & 100.00 \\
\bottomrule
\end{tabular}
\caption{Pareto distribution of extremely late post-closed resolution
updates ($\geq 1095$ days
1095 days after closure), by agency (total records: 248).}
\label{tab:extreme-postclosed-updates-pareto}
\end{table}


\subsection{Example extremely late post-closed resolution updates}

The following table presents representative examples of extremely late
post-closed updates, illustrating multi-year delays between record
closure and subsequent modification.

\begin{table}[!htbp]
\centering
\begin{tabular}{l l l r}
\toprule
Agency & Closed date & Resolution action updated date & Delay (days) \\
\midrule
DOT & 2021-08-24 13:12:10 & 2024-12-30 21:29:26 & 1{,}224.39 \\
TLC & 1899-12-31 19:00:00 & 2021-08-26 08:44:35 & 44{,}432.53 \\
DPR & 1899-12-31 19:00:00 & 2023-08-22 15:36:41 & 45{,}158.82 \\
TLC & 1899-12-31 19:00:00 & 2021-10-12 10:02:57 & 44{,}479.59 \\
DHS & 1900-01-01 00:00:00 & 2022-02-11 11:19:21 & 44{,}601.47 \\
DOT & 2021-12-02 09:58:40 & 2024-12-30 14:55:17 & 1{,}124.21 \\
TLC & 2020-05-06 16:11:07 & 2023-05-18 11:57:17 & 1{,}106.82 \\
TLC & 2020-05-13 23:40:32 & 2023-06-16 15:13:06 & 1{,}128.65 \\
TLC & 2020-05-13 15:03:28 & 2023-06-02 09:12:07 & 1{,}114.76 \\
DOT & 2021-11-29 11:36:15 & 2024-12-30 20:39:40 & 1{,}127.38 \\
\bottomrule
\end{tabular}
\caption{Illustrative examples of extremely late post-closed resolution
updates ($\geq 1095$ days
1095 days after closure), demonstrating multi-year delays.}
\label{tab:extreme-postclosed-updates-examples}
\end{table}


\subsection{\texttt{closed\_date} anomaly summary}

The following table summarizes validation checks applied to the
\texttt{closed\_date} field, evaluated using the
\texttt{America/New\_York} time zone.

\begin{table}[!htbp]
\centering
\begin{tabular}{lrr}
\toprule
Anomaly type & Count (N) & Percent (\%) \\
\midrule
Future closed dates        & 3        & 0.00 \\
Past closed dates          & 114      & 0.00 \\
Missing closed dates       & 230{,}808 & 1.44 \\
Midnight-only timestamps  & 1{,}083{,}862 & 6.77 \\
Noon-only timestamps      &   294{,}817 & 1.84 \\
Closed earlier than created &    45{,}296 & 0.28 \\
\bottomrule
\end{tabular}
\caption{Anomaly summary for the \texttt{closed\_date} field
(analysis time zone: \texttt{America/New\_York}).}
\label{tab:closed-date-anomalies}
\end{table}


\begin{table}[!htbp]
\centering
\begin{tabular}{lrrr}
\toprule
Agency & N & Pct & Cum. Pct \\
\midrule
HPD & 153{,}927 & 0.42 & 0.42 \\
DOT & 128{,}192 & 0.35 & 0.77 \\
DOB & 61{,}045 & 0.17 & 0.93 \\
DSNY & 22{,}694 & 0.06 & 1.00 \\
DEP & 604 & 0.00 & 1.00 \\
NYPD & 514 & 0.00 & 1.00 \\
DPR & 192 & 0.00 & 1.00 \\
OSE & 37 & 0.00 & 1.00 \\
DOHMH & 16 & 0.00 & 1.00 \\
DHS & 2 & 0.00 & 1.00 \\
\bottomrule
\end{tabular}
\caption{Agency distribution for \texttt{resolution\_action\_updated\_date} before \texttt{created\_date} (N = 367{,}223).}
\label{tab:resolution-action-before-created-by-agency}
\end{table}

\paragraph{Interpretation}
A large number of records (367,223) show \texttt{resolution\_action\_updated\_date} values that precede their corresponding \texttt{created\_date} timestamps, representing a logical impossibility in the workflow. This anomaly is concentrated in HPD (42\%), DOT (35\%), and DOB (17\%), together accounting for 93\% of all such cases. Unlike the 16 due date anomalies which appeared to result from a single batch processing error, this pattern spans multiple agencies and represents approximately 2.3\% of all service requests in the dataset. The systematic nature of these temporal inversions suggests either database synchronization issues during system migrations, retroactive record creation with incorrectly preserved update timestamps, or fundamental data entry problems in how resolution actions are logged relative to request initiation across multiple city agencies.



\begin{table}[!htbp]
\centering
\begin{tabular}{lr}
\toprule
Agency & Missing Closed Dates with Status \texttt{closed} \\
\midrule
DPR & 65{,}694 \\
DHS & 48{,}840 \\
DOHMH & 27{,}764 \\
TLC & 21{,}761 \\
EDC & 20{,}059 \\
DOT & 14{,}121 \\
DSNY & 10{,}579 \\
HPD & 9{,}724 \\
DEP & 7{,}430 \\
DCWP & 3{,}610 \\
\bottomrule
\end{tabular}
\caption{Top agencies with missing \texttt{closed\_date} values (total missing: 230{,}808).}
\label{tab:missing-closed-dates-by-agency}
\end{table}

\paragraph{Interpretation}
A total of 230,808 records (1.44\% of the dataset) are missing \texttt{closed\_date} values. The distribution across agencies reveals significant variation in closure documentation practices, with DPR accounting for 28\% of missing values, followed by DHS (21\%) and DOHMH (12\%). Unlike the near-universal absence of due dates, the relatively low overall rate of missing closure dates suggests that most agencies maintain consistent closure tracking. The concentration of missing values in DPR and DHS may reflect operational differences in how these agencies handle service request lifecycles, potentially indicating longer resolution timeframes, different workflow systems, or cases where formal closure is not required for certain request types.


%---------------------------------------------------------------
\section{Daylight Saving Time Analysis}

\subsection{DST Fall-back Analysis}

\begin{table}[!htbp]
\centering
\begin{tabular}{lrrrrr}
\toprule
DST End Date & N & Percentage (\%) & Cum. Pct (\%) & Avg. Negative (min) \\
\midrule
2020-11-01 & 95 & 22.04 & 22.04 & $-44.13$ \\
2021-11-07 & 86 & 19.95 & 42.00 & $-45.68$ \\
2022-11-06 & 80 & 18.56 & 60.56 & $-55.21$ \\
2023-11-05 & 104 & 24.13 & 84.69 & $-32.51$ \\
2024-11-03 & 66 & 15.31 & 100.00 & $-34.64$ \\
\midrule
TOTAL & 431 & 100.00 & {} & $-42.24$ \\
\bottomrule
\end{tabular}
\caption{Daylight Saving Time fall-back systemic negative durations by year.}
\label{tab:dst-fall-back-analysis}
\end{table}

\paragraph{Interpretation}
A total of 431 service requests exhibit systematic negative durations clustered on Daylight Saving Time fall-back dates (when clocks move back one hour at 2:00 AM). These records show an average negative duration of 42.24 minutes, consistent with timezone conversion errors where the system failed to account for the DST transition. The distribution is relatively uniform across years (2020--2024), with each fall-back date contributing 15--24\% of the total anomalies. The negative durations averaging around 40--55 minutes suggest that \texttt{closed\_date} timestamps were recorded in one timezone representation while \texttt{created\_date} used another, creating apparent temporal inversions during the hour when clocks "fall back." This systematic pattern indicates a software bug in timezone handling rather than data entry errors, and affects a small but identifiable subset of records created near the DST transition boundary.



\subsection{DST Spring-forward Analysis}

\begin{table}[!htbp]
\centering
\begin{tabular}{lrrrr}
\toprule
DST Date & Year & N & Percentage (\%) & Cumulative (\%) \\
\midrule
2020-03-08 & 2020 & 23{,}657 & 4.80 & 4.80 \\
2021-03-14 & 2021 & 54{,}492 & 11.06 & 15.86 \\
2022-03-13 & 2022 & 74{,}495 & 15.12 & 30.98 \\
2023-03-12 & 2023 & 113{,}702 & 23.08 & 54.06 \\
2024-03-10 & 2024 & 226{,}302 & 45.94 & 100.00 \\
\midrule
TOTAL & {} & 492{,}648 & 100.00 & {} \\
\bottomrule
\end{tabular}
\caption{Daylight Saving Time spring-forward systemic durations by year.}
\label{tab:dst-spring-forward-analysis}
\end{table}

\paragraph{Interpretation}
A substantial and growing number of service requests (492,648 total, representing approximately 3.1\% of all records) exhibit systematic one-hour duration anomalies clustered on Daylight Saving Time spring-forward dates (when clocks move ahead at 2:00 AM). The pattern shows a dramatic year-over-year increase, with 2024 accounting for nearly half (46\%) of all such cases compared to just 4.8\% in 2020. This exponential growth from 23,657 cases in 2020 to 226,302 in 2024 suggests either an expanding software bug affecting timezone conversions or an increasing proportion of service requests being created or closed during the DST transition window. Unlike the fall-back anomalies which produce negative durations, these spring-forward cases likely manifest as artificially inflated durations of approximately one hour, reflecting the same fundamental timezone handling deficiency but in the opposite temporal direction.


\begin{table}[!htbp]
\centering
\begin{tabular}{lrrrr}
\toprule
Year & Backlog In & Cumulative Backlog & Percentage (\%) & Cumulative (\%) \\
\midrule
2020 & 18{,}013 & 18{,}013 & 10.30 & 10.30 \\
2021 & 20{,}758 & 38{,}771 & 11.87 & 22.18 \\
2022 & 34{,}951 & 73{,}722 & 19.99 & 42.17 \\
2023 & 46{,}984 & 120{,}706 & 26.88 & 69.05 \\
2024 & 54{,}099 & 174{,}805 & 30.95 & 100.00 \\
\midrule
TOTAL & 174{,}805 & {} & 100.00 & {} \\
\bottomrule
\end{tabular}
\caption{Annual backlog of DST spring-forward affected records (2020--2024, with 2019 baseline: 18{,}013).}
\label{tab:dst-spring-forward-backlog}
\end{table}

\paragraph{Interpretation}
Beyond the 492,648 records created directly on DST spring-forward dates, an additional 174,805 service requests represent a persistent backlog of DST-affected cases that accumulated over time. Starting with a baseline of 18,013 problematic records from 2019, this backlog grew steadily each year, reaching a cumulative total of 174,805 by the end of 2024. The annual increment shows accelerating growth: from approximately 18,000--21,000 new backlog cases in 2020--2021 to 34,000--54,000 in 2022--2024, with 2024 alone contributing 31\% of the total accumulated backlog. This pattern suggests that DST-related timezone conversion issues are not being systematically corrected, but rather are compounding year over year, with newer cases adding to rather than replacing older problematic records in the system.


%---------------------------------------------------------------
\section{Analyzing Temporal Patterns}

\begin{table}[!htbp]
\centering
\begin{tabular}{rrrr}
\toprule
Hour & N & Percentage (\%) & Cumulative (\%) \\
\midrule
08 & 476{,}578 & 3.00 & 3.00 \\
09 & 645{,}220 & 4.06 & 7.05 \\
10 & 700{,}917 & 4.41 & 11.46 \\
11 & 693{,}151 & 4.36 & 15.81 \\
12 & 1{,}109{,}568 & 6.97 & 22.78 \\
13 & 659{,}930 & 4.15 & 26.93 \\
14 & 574{,}671 & 3.61 & 30.54 \\
15 & 481{,}661 & 3.03 & 33.57 \\
16 & 466{,}186 & 2.93 & 36.50 \\
17 & 484{,}071 & 3.04 & 39.54 \\
18 & 472{,}274 & 2.97 & 42.51 \\
19 & 433{,}741 & 2.73 & 45.24 \\
20 & 467{,}058 & 2.94 & 48.17 \\
21 & 557{,}980 & 3.51 & 51.68 \\
22 & 647{,}524 & 4.07 & 55.75 \\
23 & 480{,}599 & 3.02 & 58.77 \\
00 & 4{,}290{,}107 & 26.96 & 85.73 \\
01 & 506{,}637 & 3.18 & 88.92 \\
02 & 415{,}938 & 2.61 & 91.53 \\
03 & 293{,}694 & 1.85 & 93.38 \\
04 & 242{,}730 & 1.53 & 94.90 \\
05 & 240{,}867 & 1.51 & 96.41 \\
06 & 275{,}979 & 1.73 & 98.15 \\
07 & 294{,}534 & 1.85 & 100.00 \\
\bottomrule
\end{tabular}
\caption{Hourly distribution of \texttt{resolution\_action\_updated\_date} timestamps.}
\label{tab:resolution-action-hourly-pattern}
\end{table}

\paragraph{Interpretation}
The \texttt{resolution\_action\_updated\_date} field exhibits the most dramatic temporal anomaly of all date fields, with an extreme concentration at midnight (hour 00) accounting for 4.29 million records or 27\% of all entries—more than six times the expected frequency under uniform distribution and nearly three times the midnight spike observed in \texttt{closed\_date} (10\%). This massive midnight concentration, combined with a secondary peak at noon (hour 12, 7.0\%), strongly suggests systematic batch processing or automated system operations rather than human-driven updates. Business hours (08:00--23:00) show relatively uniform moderate activity (3.0--4.4\% excluding the noon spike), while overnight hours (01:00--07:00) exhibit consistently low activity (1.5--3.2\%), as expected. The midnight concentration likely represents automated resolution closure processes, scheduled database updates, or default timestamp assignments for cases where precise update times were not captured, indicating that approximately one-quarter of all resolution actions reflect system operations rather than actual staff activity times.

\begin{table}[!htbp]
\centering
\begin{tabular}{rrrr}
\toprule
Hour & N & Percentage (\%) & Cumulative (\%) \\
\midrule
08 & 773{,}681 & 4.83 & 4.83 \\
09 & 950{,}259 & 5.93 & 10.76 \\
10 & 961{,}596 & 6.00 & 16.77 \\
11 & 952{,}897 & 5.95 & 22.72 \\
12 & 936{,}427 & 5.85 & 28.56 \\
13 & 877{,}267 & 5.48 & 34.04 \\
14 & 889{,}396 & 5.55 & 39.59 \\
15 & 849{,}373 & 5.30 & 44.90 \\
16 & 828{,}336 & 5.17 & 50.07 \\
17 & 770{,}001 & 4.81 & 54.88 \\
18 & 743{,}060 & 4.64 & 59.51 \\
19 & 722{,}544 & 4.51 & 64.03 \\
20 & 733{,}756 & 4.58 & 68.61 \\
21 & 794{,}084 & 4.96 & 73.56 \\
22 & 841{,}964 & 5.26 & 78.82 \\
23 & 745{,}371 & 4.65 & 83.48 \\
00 & 611{,}736 & 3.82 & 87.29 \\
01 & 407{,}753 & 2.55 & 89.84 \\
02 & 272{,}353 & 1.70 & 91.54 \\
03 & 196{,}263 & 1.23 & 92.77 \\
04 & 170{,}167 & 1.06 & 93.83 \\
05 & 189{,}998 & 1.19 & 95.02 \\
06 & 295{,}051 & 1.84 & 96.86 \\
07 & 503{,}367 & 3.14 & 100.00 \\
\bottomrule
\end{tabular}
\caption{Hourly distribution of \texttt{created\_date} timestamps.}
\label{tab:created-date-hourly-pattern}
\end{table}

\paragraph{Interpretation}
The \texttt{created\_date} field exhibits a markedly different temporal pattern than \texttt{resolution\_action\_updated\_date} and \texttt{closed\_date}, displaying a distribution much more consistent with genuine human activity cycles. Business hours (08:00--23:00) account for 83.5\% of all service request creation, with peak activity concentrated in morning hours (9--11 AM at 5.9--6.0\% each) when citizens are most likely to report issues. The distribution remains relatively uniform throughout the workday (4.5--6.0\%) with a gradual decline into evening hours (17:00--23:00 dropping from 4.8\% to 4.7\%). Notably, unlike the resolution and closure fields which show pronounced midnight spikes of 27\% and 10\% respectively, the created date shows only 3.8\% at midnight—slightly elevated but far more reasonable. Overnight hours (01:00--07:00) appropriately show minimal activity (1.1--3.1\%), following expected patterns of public service request behavior and confirming that service request creation timestamps capture actual user behavior rather than system artifacts or batch processing operations.


\begin{table}[!htbp]
\centering
\begin{tabular}{rrrr}
\toprule
Hour & N & Percentage (\%) & Cumulative (\%) \\
\midrule
08 & 649{,}781 & 4.12 & 4.12 \\
09 & 826{,}943 & 5.24 & 9.35 \\
10 & 860{,}358 & 5.45 & 14.80 \\
11 & 838{,}185 & 5.31 & 20.11 \\
12 & 1{,}061{,}503 & 6.72 & 26.84 \\
13 & 831{,}712 & 5.27 & 32.11 \\
14 & 770{,}651 & 4.88 & 36.99 \\
15 & 666{,}224 & 4.22 & 41.21 \\
16 & 655{,}190 & 4.15 & 45.36 \\
17 & 686{,}065 & 4.35 & 49.71 \\
18 & 681{,}544 & 4.32 & 54.02 \\
19 & 619{,}046 & 3.92 & 57.95 \\
20 & 719{,}339 & 4.56 & 62.50 \\
21 & 637{,}714 & 4.04 & 66.54 \\
22 & 665{,}851 & 4.22 & 70.76 \\
23 & 487{,}830 & 3.09 & 73.85 \\
00 & 1{,}615{,}329 & 10.23 & 84.08 \\
01 & 507{,}051 & 3.21 & 87.30 \\
02 & 460{,}986 & 2.92 & 90.22 \\
03 & 297{,}199 & 1.88 & 92.10 \\
04 & 242{,}972 & 1.54 & 93.64 \\
05 & 258{,}603 & 1.64 & 95.28 \\
06 & 322{,}297 & 2.04 & 97.32 \\
07 & 423{,}519 & 2.68 & 100.00 \\
\bottomrule
\end{tabular}
\caption{Hourly distribution of \texttt{closed\_date} timestamps.}
\label{tab:closed-date-hourly-pattern}
\end{table}

\paragraph{Interpretation}
The \texttt{closed\_date} field exhibits a hybrid pattern combining human activity with automated processing. Business hours (08:00--23:00) show reasonable activity consistent with staff closure operations (3.1--6.7\%), with peaks during traditional work hours (9--11 AM at 5.2--5.5\%) and a pronounced spike at noon (6.7\%). The most striking feature is the midnight concentration of 1.62 million records (10.2\%), suggesting batch closure processes or automated system operations that finalize cases at day boundaries, though this is less extreme than the 27\% midnight spike observed in \texttt{resolution\_action\_updated\_date}. Daytime activity remains relatively sustained through evening hours (20:00--22:00 at 4.0--4.6\%), indicating that closure activities extend beyond standard business hours. Overnight hours (01:00--07:00) maintain low but non-negligible activity (1.5--3.2\%), likely reflecting both after-hours field work completions and continued automated processing throughout the night.


%---------------------------------------------------------------
\section{Duration Issue}

\subsection{Positive Durations}
\begin{table}[!htbp]
\centering
\begin{tabular}{lr}
\toprule
Metric & Value \\
\midrule
Total records & 15{,}353{,}267 \\
Minimum (days) & 0.0000 \\
Maximum (days) & 3{,}929.4369 \\
Mean (days) & 23.7231 \\
Median (days) & 0.4389 \\
Standard deviation (days) & 121.4013 \\
\bottomrule
\end{tabular}
\caption{Summary statistics for positive service request durations (\texttt{closed\_date} $-$ \texttt{created\_date}).}
\label{tab:positive-duration-summary}
\end{table}


\subsection{Negative Durations}
\begin{table}[!htbp]
\centering
\begin{tabular}{lr}
\toprule
Metric & Value \\
\midrule
Total records & 45{,}091 \\
Minimum (days) & $-364.9319$ \\
Maximum (days) & $-0.0000$ \\
Mean (days) & $-6.0294$ \\
Median (days) & $-2.8910$ \\
Standard deviation (days) & 22.0088 \\
\bottomrule
\end{tabular}
\caption{Summary statistics for negative service request durations (closure before creation).}
\label{tab:negative-duration-summary}
\end{table}

\paragraph{Interpretation}
The vast majority of service requests (15.35 million, or 99.7\%) exhibit positive durations, with closure dates following creation dates as logically expected. However, the positive duration distribution is highly right-skewed, with a median of just 0.44 days (10.5 hours) compared to a mean of 23.7 days, indicating that while most requests resolve quickly, a small subset remains open for extended periods (maximum: 3,929 days or 10.8 years). The standard deviation of 121 days reflects this extreme variability. A concerning 45,091 records (0.29\%) display the logical impossibility of negative durations, where closure dates precede creation dates. These negative durations average $-6.0$ days with a median of $-2.9$ days, ranging up to $-365$ days, suggesting systematic data entry errors, database synchronization failures, or timezone conversion issues rather than isolated mistakes. The concentration of negative durations in the 2--6 day range indicates potential batch processing problems or retroactive record creation with incorrectly assigned timestamps.



\subsection{Short Duration Analysis and Threshold Selection}

\begin{table}[!htbp]
\centering
\begin{tabular}{lrr}
\toprule
Dataset Stage & N & Percentage (\%) \\
\midrule
Original positive durations & 15{,}353{,}267 & 100.00 \\
After minimum cutoff (2 sec) & 15{,}349{,}807 & 99.98 \\
After maximum cutoff (172{,}800 sec / 2 days) & 9{,}961{,}142 & 64.89 \\
\bottomrule
\end{tabular}
\caption{Duration filtering stages for outlier analysis.}
\label{tab:duration-filtering-stages}
\end{table}

\begin{table}[!htbp]
\centering
\begin{tabular}{lrr}
\toprule
Statistic & Raw Data (sec) & Truncated Data (sec) \\
\midrule
Mean & 2{,}050{,}138 & 27{,}082 \\
Median & 37{,}996 & 5{,}220 \\
Standard deviation & 10{,}490{,}210 & 42{,}642 \\
Mean/Median ratio & 53.96 & 5.19 \\
\bottomrule
\end{tabular}
\caption{Distribution comparison before and after truncation at 2 days. Mean/Median ratio indicates degree of right-skew (1.0 = normal, $>$1.0 = right-skewed).}
\label{tab:duration-distribution-comparison}
\end{table}

\begin{table}[!htbp]
\centering
\begin{tabular}{lrrr}
\toprule
Method & Threshold (sec) & Outlier Count & Outlier (\%) \\
\midrule
LogNormal 3SD & 27.95 & 10{,}231 & 0.07 \\
Percentile 1\% & 95.00 & 98{,}476 & 0.64 \\
LogNormal 2SD & 173.78 & 200{,}604 & 1.31 \\
Percentile 5\% & 417.00 & 497{,}725 & 3.24 \\
Percentile 10\% & 800.00 & 995{,}604 & 6.49 \\
Truncated 2SD & --- & 0 & 0.00 \\
Truncated 3SD & --- & 0 & 0.00 \\
MAD 2x & --- & 0 & 0.00 \\
MAD 3x & --- & 0 & 0.00 \\
IQR 1.5 & --- & 0 & 0.00 \\
IQR 3.0 & --- & 0 & 0.00 \\
\bottomrule
\end{tabular}
\caption{Outlier detection methods applied to truncated duration data. Methods yielding zero outliers could not identify a meaningful threshold due to extreme skewness.}
\label{tab:outlier-detection-methods}
\end{table}

\begin{table}[!htbp]
\centering
\begin{tabular}{lr}
\toprule
Metric & Value \\
\midrule
Total observations (2--90 sec) & 93{,}408 \\
Mean (sec) & 55 \\
Mean (days) & 0.000638 \\
Median (sec) & 57 \\
Median (days) & 0.000660 \\
Standard deviation (sec) & 21 \\
Standard deviation (days) & 0.000247 \\
LogNormal 3SD threshold (sec) & 28 \\
\bottomrule
\end{tabular}
\caption{Summary statistics for very short durations (2--90 seconds).}
\label{tab:very-short-duration-summary}
\end{table}

\paragraph{Interpretation}
The duration data exhibits extreme right-skewness, with a mean/median ratio of 53.96 in the raw data, indicating that a small number of extremely long-duration cases dominate the distribution. After truncating at 2 days (172,800 seconds) to focus on short-duration analysis, approximately 35\% of records were excluded, and the mean/median ratio improved to 5.19, though still indicating substantial skew. Traditional outlier detection methods based on standard deviations, median absolute deviation (MAD), and interquartile range (IQR) failed to identify any thresholds, returning zero outliers due to the extreme concentration of values at the low end of the distribution. Only log-normal and percentile-based methods successfully identified potential short-duration thresholds. The LogNormal 3SD method suggests a threshold of 28 seconds, below which lie 10,231 records (0.07\%)—cases that are statistically unusual even accounting for the log-normal distribution characteristic of administrative process times. Among the 93,408 records with durations between 2--90 seconds, the median of 57 seconds represents cases resolved in under one minute, likely reflecting automated closures, duplicate entries, or requests immediately identified as misdirected rather than genuine service delivery.


\begin{table}[!htbp]
\centering
\begin{tabular}{lrr}
\toprule
Duration Category & N & Percentage (\%) \\
\midrule
\textbf{Total closed records (denominator)} & \textbf{15{,}785{,}892} & \textbf{100.00} \\
\addlinespace
\textit{Negative durations (closure before creation):} & & \\
\quad Negative total & 45{,}296 & 0.29 \\
\quad \quad Small ($>-$549 days) & 45{,}180 & 0.29 \\
\quad \quad Large ($-$549 to $-$857 days) & 1 & 0.00 \\
\quad \quad Extreme ($<-$857 days) & 115 & 0.00 \\
\addlinespace
\textit{Zero and near-zero durations:} & & \\
\quad Zero duration (same timestamp) & 387{,}329 & 2.45 \\
\quad Near-zero ($\leq$28 sec, excluding zero) & 14{,}741 & 0.09 \\
\quad Exactly 1 second & 3{,}460 & 0.02 \\
\addlinespace
\textit{Positive durations:} & & \\
\quad Positive total & 15{,}353{,}267 & 97.26 \\
\quad \quad Small ($\leq$730 days) & 15{,}212{,}398 & 96.37 \\
\quad \quad Large (730--1{,}826 days) & 139{,}628 & 0.88 \\
\quad \quad Extreme ($>$1{,}826 days) & 1{,}240 & 0.01 \\
\bottomrule
\end{tabular}
\caption{Duration category distribution for closed service requests with non-missing \texttt{created\_date}.}
\label{tab:duration-category-summary}
\end{table}

\begin{table}[!htbp]
\centering
\begin{tabular}{lrrrr}
\toprule
Category & Mean (days) & Min (days) & Max (days) & N \\
\midrule
\multicolumn{5}{l}{\textit{Negative durations:}} \\
\quad All negative & $-118.71$ & $-44{,}601.59$ & 0.00 & 45{,}296 \\
\quad Small negative & $-6.87$ & $-548.76$ & 0.00 & 45{,}180 \\
\quad Large negative & $-856.56$ & $-856.56$ & $-856.56$ & 1 \\
\quad Extreme negative & $-44{,}050.91$ & $-44{,}601.59$ & $-5{,}545.43$ & 115 \\
\addlinespace
\multicolumn{5}{l}{\textit{Positive durations:}} \\
\quad All positive & 23.72 & 0.00 & 3{,}929.44 & 15{,}353{,}267 \\
\quad Small positive & 13.70 & 0.00 & 730.50 & 15{,}212{,}398 \\
\quad Large positive & 1{,}098.94 & 730.54 & 1{,}826.21 & 139{,}628 \\
\quad Extreme positive & 1{,}875.98 & 1{,}826.41 & 2{,}074.85 & 1{,}240 \\
\bottomrule
\end{tabular}
\caption{Duration statistics by category (days).}
\label{tab:duration-statistics-by-category}
\end{table}

\paragraph{Interpretation}
Service request durations exhibit a trimodal quality pattern dominated by appropriate positive durations (97.3\%) but with concerning anomalies at both extremes. Zero-duration records represent 2.45\% of cases (387,329), likely reflecting same-timestamp creation and closure from automated processing or immediate rejection of invalid requests. Negative durations affect 0.29\% of records (45,296), split between "small" violations averaging $-6.9$ days (99.7\% of negative cases) and 116 extreme outliers averaging $-44,051$ days (122 years)—the latter representing catastrophic data entry errors with minimum values reaching $-44,602$ days. Among positive durations, the distribution is highly right-skewed: small durations ($\leq$2 years) average just 13.7 days and comprise 96.4\% of all records, while 1,240 extreme cases exceed 5 years, with the maximum reaching 3,929 days (10.8 years). The 14,741 near-zero positive durations (under 28 seconds) likely capture automated or instantaneous resolutions. This pattern suggests that while the 311 system generally produces valid duration data, systematic issues exist with timestamp synchronization, particularly affecting a small subset of records with biologically or operationally impossible timescales.


\begin{table}[!htbp]
\centering
\begin{tabular}{lllr}
\toprule
Agency & Created Date & Closed Date & Duration (min) \\
\midrule
DOT & 2020-06-04 09:16:00 & 2020-05-21 09:16:00 & $-20{,}160$ \\
DOT & 2021-01-04 14:06:00 & 2020-12-30 14:06:00 & $-7{,}200$ \\
DOT & 2020-01-27 13:52:00 & 2020-01-24 13:52:00 & $-4{,}320$ \\
DOT & 2020-05-01 11:55:00 & 2020-04-28 11:55:00 & $-4{,}320$ \\
DOT & 2020-07-27 10:09:00 & 2020-07-24 10:09:00 & $-4{,}320$ \\
DOT & 2021-05-03 13:45:00 & 2021-04-30 13:45:00 & $-4{,}320$ \\
DOT & 2020-04-27 09:11:00 & 2020-04-24 20:45:00 & $-3{,}626$ \\
DOT & 2021-03-17 07:53:00 & 2021-03-16 00:50:00 & $-1{,}863$ \\
DOT & 2021-07-02 08:29:00 & 2021-07-01 08:00:00 & $-1{,}469$ \\
DOT & 2021-05-05 08:58:00 & 2021-05-04 08:57:00 & $-1{,}441$ \\
\bottomrule
\end{tabular}
\caption{Sample of negative duration records (random 10 of 45,296, all from DOT).}
\label{tab:negative-duration-examples}
\end{table}

\paragraph{Interpretation}
This random sample of negative durations reveals a systematic pattern concentrated in the Department of Transportation (DOT), with durations predominantly clustering at exact 3-day ($-4{,}320$ minutes) and 14-day ($-20{,}160$ minutes) intervals. The precision of these intervals—matching exactly to the minute in four cases—suggests automated batch processing errors rather than random data entry mistakes. The temporal pattern shows closure dates consistently preceding creation dates by 1--14 days, with the majority falling in the 1--3 day range. These cases likely represent service requests that were retroactively entered into the system with incorrectly preserved or assigned closure timestamps, possibly during database migrations, system updates, or when historical paper records were digitized with incorrect date sequencing.



\begin{table}[!htbp]
\centering
\begin{tabular}{lllr}
\toprule
Agency & Created Date & Closed Date & Duration (sec) \\
\midrule
DOB & 2020-05-19 12:53:35 & 2020-05-19 12:53:35 & 0 \\
DOHMH & 2020-10-03 15:22:58 & 2020-10-03 15:22:58 & 0 \\
DOT & 2021-03-10 10:00:00 & 2021-03-10 10:00:00 & 0 \\
DOHMH & 2021-06-04 09:18:06 & 2021-06-04 09:18:06 & 0 \\
DEP & 2021-06-30 17:06:00 & 2021-06-30 17:06:00 & 0 \\
DEP & 2021-07-11 09:25:00 & 2021-07-11 09:25:00 & 0 \\
DOT & 2024-04-06 18:48:00 & 2024-04-06 18:48:00 & 0 \\
DOHMH & 2024-07-15 09:11:24 & 2024-07-15 09:11:24 & 0 \\
DOHMH & 2024-08-15 19:06:09 & 2024-08-15 19:06:09 & 0 \\
DOHMH & 2024-12-10 09:19:25 & 2024-12-10 09:19:25 & 0 \\
\bottomrule
\end{tabular}
\caption{Sample of zero-duration records with identical creation and closure timestamps (random 10 of 387,329).}
\label{tab:zero-duration-examples}
\end{table}

\paragraph{Interpretation}
Zero-duration records, representing 2.45\% of all closed service requests (387,329 cases), exhibit identical \texttt{created\_date} and \texttt{closed\_date} timestamps down to the second. This sample reveals diverse agency representation (DOHMH, DOT, DEP, DOB) and spans the entire dataset timeframe (2020--2024), suggesting this is a persistent systematic pattern rather than an isolated incident. The timestamps show varying levels of precision—some with full seconds (12:53:35) and others rounded to the minute (10:00:00)—indicating multiple mechanisms producing zero durations. These cases likely represent: (1) automated immediate closures of duplicate or invalid requests, (2) batch processing where creation and closure occur in a single transaction, (3) requests that were pre-screened and rejected before formal assignment, or (4) data entry workflows where staff create and immediately close cases for record-keeping purposes without actual service delivery occurring.


\begin{table}[!htbp]
\centering
\begin{tabular}{lllr}
\toprule
Agency & Created Date & Closed Date & Duration (sec) \\
\midrule
EDC & 2020-01-04 19:45:01 & 2020-01-04 19:45:02 & 1 \\
EDC & 2020-01-17 14:41:31 & 2020-01-17 14:41:32 & 1 \\
DPR & 2020-10-06 22:24:56 & 2020-10-06 22:24:57 & 1 \\
DOHMH & 2021-02-03 00:00:00 & 2021-02-03 00:00:01 & 1 \\
DOT & 2021-03-20 14:27:36 & 2021-03-20 14:27:37 & 1 \\
DOHMH & 2021-08-12 00:00:00 & 2021-08-12 00:00:01 & 1 \\
DOHMH & 2023-08-04 00:00:00 & 2023-08-04 00:00:01 & 1 \\
DOHMH & 2024-07-22 00:00:00 & 2024-07-22 00:00:01 & 1 \\
DOHMH & 2024-07-31 00:00:00 & 2024-07-31 00:00:01 & 1 \\
DOHMH & 2024-11-29 00:00:00 & 2024-11-29 00:00:01 & 1 \\
\bottomrule
\end{tabular}
\caption{Sample of one-second duration records (random 10 of 3,460).}
\label{tab:one-second-duration-examples}
\end{table}

\paragraph{Interpretation}
One-second durations, while rare (0.02\% of closed requests or 3,460 cases), reveal a distinctive pattern suggesting automated processing rather than human activity. Six of the ten sampled records from DOHMH show a characteristic midnight signature (00:00:00 to 00:00:01), indicating batch operations that create and immediately close records within the same second during automated overnight processing. The remaining records from EDC, DPR, and DOT display varied timestamps throughout the day with precise second-level granularity (e.g., 19:45:01 to 19:45:02), suggesting rapid automated validation or rejection workflows. The consistency of exactly one-second intervals—rather than a distribution of very short durations—points to programmatic closure logic, possibly representing: (1) automated duplicate detection systems, (2) instant rejection of malformed requests, (3) system-generated test records, or (4) batch imports from external systems where creation and closure are recorded as sequential database operations within the same transaction cycle.


\begin{table}[!htbp]
\centering
\begin{tabular}{lllr}
\toprule
Agency & Created Date & Closed Date & Duration (sec) \\
\midrule
DOHMH & 2022-03-16 00:00:00 & 2022-03-16 00:00:01 & 1 \\
EDC & 2020-02-19 07:54:44 & 2020-02-19 07:54:46 & 2 \\
NYPD & 2021-07-20 04:11:55 & 2021-07-20 04:12:13 & 18 \\
NYPD & 2020-05-19 13:07:17 & 2020-05-19 13:07:40 & 23 \\
NYPD & 2020-08-08 22:27:34 & 2020-08-08 22:27:57 & 23 \\
NYPD & 2021-04-24 22:48:16 & 2021-04-24 22:48:39 & 23 \\
NYPD & 2021-07-20 02:00:18 & 2021-07-20 02:00:44 & 26 \\
NYPD & 2024-01-26 22:56:11 & 2024-01-26 22:56:37 & 26 \\
NYPD & 2020-04-24 21:37:32 & 2020-04-24 21:37:59 & 27 \\
NYPD & 2024-09-15 13:09:09 & 2024-09-15 13:09:37 & 28 \\
\bottomrule
\end{tabular}
\caption{Sample of suspicious/questionable duration records (2--28 seconds, random 10 of 14,741).}
\label{tab:near-zero-duration-examples}
\end{table}

\paragraph{Interpretation}
Suspicious or questionable durations between 2--28 seconds (14,741 cases, 0.09\% of closed requests) represent statistically suspicious cases identified by the LogNormal 3SD threshold. This sample reveals a striking concentration in NYPD (8 of 10 records), with durations clustering at 23 and 26 seconds—intervals too consistent to reflect natural human processing times but too variable to be simple automated operations. The timestamps span various hours including late night (02:00, 04:11) and evening (21:37, 22:27), suggesting these are not batch-processed at standard intervals like midnight operations. These sub-30-second resolutions likely represent: (1) automated validation systems that immediately reject malformed or duplicate requests after brief processing, (2) service requests that were opened in error and immediately canceled by the requester or system operator, (3) pre-screening workflows where cases are instantly routed or rejected based on automated classification, or (4) NYPD-specific rapid triage systems that can assess and close certain request types (possibly noise complaints or non-emergency reports) within seconds of receipt.



\begin{table}[!htbp]
\centering
\begin{tabular}{lllr}
\toprule
Agency & Created Date & Closed Date & Duration (days) \\
\midrule
DPR & 2020-09-08 14:37:02 & 2025-01-14 13:38:38 & 1{,}589.00 \\
DOHMH & 2020-06-02 11:39:02 & 2024-06-03 14:34:18 & 1{,}462.12 \\
DOHMH & 2020-06-30 09:48:33 & 2024-06-03 14:48:34 & 1{,}434.21 \\
DPR & 2021-07-11 19:15:28 & 2025-04-22 17:55:52 & 1{,}380.94 \\
EDC & 2021-03-06 15:41:34 & 2024-10-23 12:35:06 & 1{,}326.83 \\
DOHMH & 2020-12-05 13:49:33 & 2024-06-03 16:01:53 & 1{,}276.05 \\
DPR & 2021-04-19 11:09:43 & 2024-09-24 10:59:27 & 1{,}253.99 \\
DPR & 2022-05-24 11:18:26 & 2024-11-12 16:30:28 & 903.26 \\
DOHMH & 2022-01-23 08:44:53 & 2024-06-03 19:18:08 & 862.40 \\
DOHMH & 2022-04-14 09:44:21 & 2024-06-03 20:01:25 & 781.43 \\
\bottomrule
\end{tabular}
\caption{Sample of large positive duration records (730--1{,}826 days, random 10 of 139,628).}
\label{tab:large-positive-duration-examples}
\end{table}

\begin{table}[!htbp]
\centering
\begin{tabular}{lrrr}
\toprule
Agency & N & Percentage & Cumulative (\%) \\
\midrule
DPR & 56{,}372 & 0.40 & 0.40 \\
DOHMH & 42{,}800 & 0.31 & 0.71 \\
EDC & 32{,}529 & 0.23 & 0.94 \\
DSNY & 2{,}816 & 0.02 & 0.96 \\
DOB & 2{,}236 & 0.02 & 0.98 \\
DOT & 1{,}727 & 0.01 & 0.99 \\
HPD & 418 & 0.00 & 0.99 \\
DEP & 351 & 0.00 & 1.00 \\
DOE & 189 & 0.00 & 1.00 \\
TLC & 142 & 0.00 & 1.00 \\
DHS & 25 & 0.00 & 1.00 \\
NYPD & 20 & 0.00 & 1.00 \\
\bottomrule
\end{tabular}
\caption{Agency distribution for large positive durations (730--1{,}826 days, N = 139{,}628).}
\label{tab:large-positive-duration-by-agency}
\end{table}

\paragraph{Interpretation}
Large positive durations between 2--5 years (139,628 cases, 0.88\% of closed requests) represent service requests that remained open for extended periods before resolution. The agency distribution reveals extreme concentration: DPR accounts for 40\% of these cases (56,372), DOHMH for 31\% (42,800), and EDC for 23\% (32,529), together comprising 94\% of all multi-year durations. The sample shows durations ranging from 781 days (2.1 years) to 1,589 days (4.4 years), with several cases closing in June 2024 or later despite creation dates in 2020--2021. This temporal pattern suggests either: (1) capital projects or infrastructure work requiring multi-year timelines (particularly for DPR parks maintenance and EDC economic development initiatives), (2) complex regulatory or permitting processes with extended review periods (DOHMH health inspections or violations), (3) cases left open inadvertently and closed during periodic system cleanups (note the June 2024 closure clustering), or (4) legitimate long-term monitoring situations where the request remains technically open while work progresses through multiple phases.




\begin{table}[!htbp]
\centering
\begin{tabular}{lllr}
\toprule
Agency & Created Date & Closed Date & Duration (days) \\
\midrule
DPR & 2020-01-07 16:05:01 & 2025-04-22 17:02:52 & 1{,}932.00 \\
DPR & 2020-01-12 14:24:54 & 2025-04-22 17:02:59 & 1{,}927.07 \\
DPR & 2020-01-31 11:24:07 & 2025-04-22 17:03:33 & 1{,}908.19 \\
DPR & 2020-01-31 20:59:07 & 2025-04-22 17:03:43 & 1{,}907.79 \\
DPR & 2020-03-13 18:07:03 & 2025-04-22 17:08:57 & 1{,}865.96 \\
DPR & 2020-03-18 10:45:09 & 2025-04-22 17:09:15 & 1{,}861.27 \\
DPR & 2020-03-29 20:07:52 & 2025-04-22 17:10:04 & 1{,}849.88 \\
DPR & 2020-04-05 14:17:35 & 2025-04-22 17:10:19 & 1{,}843.12 \\
DPR & 2020-04-06 15:53:53 & 2025-04-22 17:10:21 & 1{,}842.05 \\
DPR & 2020-04-10 13:08:27 & 2025-04-22 17:10:38 & 1{,}838.17 \\
\bottomrule
\end{tabular}
\caption{Sample of extreme positive duration records ($>$1{,}826 days or 5 years, random 10 of 1{,}240, all from DPR).}
\label{tab:extreme-positive-duration-examples}
\end{table}

\begin{table}[!htbp]
\centering
\begin{tabular}{lrrr}
\toprule
Agency & N & Percentage & Cumulative (\%) \\
\midrule
DPR & 1{,}162 & 0.94 & 0.94 \\
DOHMH & 33 & 0.03 & 0.96 \\
DOB & 17 & 0.01 & 0.98 \\
DSNY & 10 & 0.01 & 0.99 \\
DEP & 9 & 0.01 & 0.99 \\
DOT & 8 & 0.01 & 1.00 \\
\bottomrule
\end{tabular}
\caption{Agency distribution for extreme positive durations ($>$1{,}826 days, N = 1{,}240).}
\label{tab:extreme-positive-duration-by-agency}
\end{table}

\paragraph{Interpretation}
Extreme positive durations exceeding 5 years (1,240 cases, 0.01\% of closed requests) represent the longest-running service requests in the dataset and are overwhelmingly concentrated in a single agency and temporal pattern. DPR accounts for 94\% of all extreme durations (1,162 cases), with the remaining 78 cases scattered across five other agencies. The sample reveals a striking signature: all ten records were created in early 2020 (January--April) and closed on the exact same date—April 22, 2025—with closure timestamps clustering within a 7-minute window (17:02:52 to 17:10:38). This pattern strongly suggests a mass cleanup operation rather than genuine 5-year service delivery timelines. The durations range from 1,838 to 1,932 days (5.0--5.3 years), likely representing service requests that were opened during the early COVID-19 pandemic period when parks operations were disrupted, then left dormant in the system until a systematic batch closure event in April 2025 cleared out these aged cases. The concentration in DPR and the pandemic-era creation dates suggest these may be facility maintenance or capital improvement requests that were deprioritized during the public health emergency and only recently addressed through a comprehensive backlog resolution initiative.


\begin{table}[!htbp]
\centering
\begin{tabular}{lllr}
\toprule
Agency & Created Date & Closed Date & Duration (days) \\
\midrule
DPR & 2020-01-07 16:05:01 & 2025-04-22 17:02:52 & 1{,}932.00 \\
DPR & 2020-01-12 14:24:54 & 2025-04-22 17:02:59 & 1{,}927.07 \\
DPR & 2020-01-31 11:24:07 & 2025-04-22 17:03:33 & 1{,}908.19 \\
DPR & 2020-01-31 20:59:07 & 2025-04-22 17:03:43 & 1{,}907.79 \\
DPR & 2020-03-13 18:07:03 & 2025-04-22 17:08:57 & 1{,}865.96 \\
DPR & 2020-03-18 10:45:09 & 2025-04-22 17:09:15 & 1{,}861.27 \\
DPR & 2020-03-29 20:07:52 & 2025-04-22 17:10:04 & 1{,}849.88 \\
DPR & 2020-04-05 14:17:35 & 2025-04-22 17:10:19 & 1{,}843.12 \\
DPR & 2020-04-06 15:53:53 & 2025-04-22 17:10:21 & 1{,}842.05 \\
DPR & 2020-04-10 13:08:27 & 2025-04-22 17:10:38 & 1{,}838.17 \\
\bottomrule
\end{tabular}
\caption{Sample of extreme positive duration records ($>$1{,}826 days or 5 years, random 10 of 1{,}240, all from DPR).}
\label{tab:extreme-positive-duration-examples}
\end{table}

\begin{table}[!htbp]
\centering
\begin{tabular}{lrrr}
\toprule
Agency & N & Percentage & Cumulative (\%) \\
\midrule
DPR & 1{,}162 & 0.94 & 0.94 \\
DOHMH & 33 & 0.03 & 0.96 \\
DOB & 17 & 0.01 & 0.98 \\
DSNY & 10 & 0.01 & 0.99 \\
DEP & 9 & 0.01 & 0.99 \\
DOT & 8 & 0.01 & 1.00 \\
\bottomrule
\end{tabular}
\caption{Agency distribution for extreme positive durations ($>$1{,}826 days, N = 1{,}240).}
\label{tab:extreme-positive-duration-by-agency}
\end{table}

\paragraph{Interpretation}
Extreme positive durations exceeding 5 years (1,240 cases, 0.01\% of closed requests) represent the longest-running service requests in the dataset and are overwhelmingly concentrated in a single agency and temporal pattern. DPR accounts for 94\% of all extreme durations (1,162 cases), with the remaining 78 cases scattered across five other agencies. The sample reveals a striking signature: all ten records were created in early 2020 (January--April) and closed on the exact same date—April 22, 2025—with closure timestamps clustering within a 7-minute window (17:02:52 to 17:10:38). This pattern strongly suggests a mass cleanup operation rather than genuine 5-year service delivery timelines. The durations range from 1,838 to 1,932 days (5.0--5.3 years), likely representing service requests that were opened during the early COVID-19 pandemic period when parks operations were disrupted, then left dormant in the system until a systematic batch closure event in April 2025 cleared out these aged cases. The concentration in DPR and the pandemic-era creation dates suggest these may be facility maintenance or capital improvement requests that were deprioritized during the public health emergency and only recently addressed through a comprehensive backlog resolution initiative.

%---------------------------------------------------------------
\section{Reproducibility and Code Access}


\subsection*{System Requirements}


\begin{itemize}
\item R version 4.5.2 or higher (\url{https://cran.r-project.org/})
\item RStudio version 2026.01.0 build 392 or higher (recommended): \\
\url{https://posit.co/download/rstudio-desktop/}
\item Minimum RAM: 16 GB (tested and verified; 32 GB provides faster execution)
\item Disk space: ~15 GB (9 GB uncompressed data + intermediate files + outputs)
\item Operating System: Tested on Windows 10/11; macOS/Linux users will need to 
modify file paths in \texttt{base\_dir}
\end{itemize}


\subsection*{Data and Code Repository}


\begin{itemize}
\item Source code: \url{https://github.com/tusseyd/nyc_311_data_cleaning}
\item NYC 311 dataset (1.57 GB zipped, 9 GB uncompressed): \\
\url{https://figshare.com/s/e8d479c391edb7224bfa}
\item USPS Zip Codes (4.5 MB): \url{https://figshare.com/s/8aea027d06f4903f2227}
\item Reference console output for validation is available in GitHub repository.
\end{itemize}

\subsection*{Required R Packages}
Both scripts automatically check for and install missing packages. The analysis 
requires: \texttt{fasttime}, 
\texttt{clock}, \texttt{zoo}, \texttt{lubridate}, \texttt{sf}, \texttt{stringr}, \texttt{stringdist}, 
\texttt{arrow}, \texttt{ggplot2}, \texttt{dplyr}, \texttt{tidyverse}, \texttt{ggpmisc}, 
\texttt{gridExtra}, \texttt{grid}, \texttt{qcc}, \texttt{qicharts2}, \texttt{gt}, \texttt{DT}, 
\texttt{bslib}, \texttt{shiny}, \texttt{httr}, \texttt{rlang}, \texttt{styler}, 
\texttt{renv}, and \texttt{data.table}.


\subsection*{Analysis Reproduction Steps}


Source files are available at:
\begin{itemize}
  \item \textsc{R} source code (GitHub): \url{https://github.com/tusseyd/nyc_311_data_cleaning}
  \item USPS Zip Code dataset (Figshare): \url{https://figshare.com/s/8aea027d06f4903f2227}
  \item NYC 311 dataset (Figshare): \url{https://figshare.com/s/e8d479c391edb7224bfa}
\end{itemize}
\vspace{0.5cm}
Step 1: Download Data Files
\begin{enumerate}
\item Download the NYC 311 and USPS Zip Code datasets from Figshare. (links above)
\item Do not rename these files as the scripts expect original filenames:
\begin{itemize}
\item \texttt{5-year\_311SR\_01-01-2020\_thru\_12-31-2024\_AS\_OF\_10-10-2025.zip}
\item \texttt{zip\_code\_database.csv}
\end{itemize}
\item Extract the CSV from the zip file (~9 GB uncompressed)
\end{enumerate}
\vspace{0.5cm}
Step 2: Set Up Project Structure
\begin{enumerate}
\item Clone or download the entire GitHub repository: \\
\url{https://github.com/tusseyd/nyc_311_data_cleaning}
\item Important: Ensure you download all files including:
\begin{itemize}
\item \texttt{code/data\_prep\_for\_jds\_datacleansing.R}
\item \texttt{code/jds\_datacleansing.R}
\item All files in \texttt{code/functions/} directory (required dependencies)
\end{itemize}
\item The data preparation script will automatically create the following directory structure:

\vspace{0.5cm}
\dirtree{%
.1 nyc\_311\_data\_cleaning/ (R Working Directory).
.2 analytics/.
.3 field\_usage\_summary\_table.csv.
.2 charts/.
.3 [82 PDF files].
.2 code/.
.3 data\_prep\_for\_jds\_datacleansing.R.
.3 jds\_datacleansing.R.
.3 functions/.
.4 [function files].
.2 console\_output/.
.3 JDS\_data\_prep\_console\_output.txt.
.3 JDS\_datacleansing\_console\_output.txt.
.2 data/.
.3 5-year\_311SR\_01-01-2020\_thru\_12-31-2024\_AS\_OF\_10-10-2025.rds.
.3 USPS\_zipcodes.rds.
.3 raw\_data/.
.4 5-year\_311SR\_01-01-2020\_thru\_12-31-2024\_AS\_OF\_10-10-2025.csv.
.4 zip\_code\_database.csv.
.2 reference\_outputs/.
.3 JDS\_data\_prep\_console\_output.txt.
.3 JDS\_datacleansing\_console\_output.txt.
}
\vspace{0.5cm}

\item Open RStudio and open \texttt{code/data\_prep\_for\_jds\_datacleansing.R}
\item Modify the \texttt{base\_dir} variable (line 85) to match your local path:
\begin{verbatim}
base_dir <- file.path("your", "path", "here", 
                      "nyc_311_data_cleaning")
\end{verbatim}
\textit{Note: Non-Windows users should use forward slashes or platform-appropriate path separators}
\item Place the downloaded data files:
\begin{itemize}
\item \texttt{5-year\_311SR\_01-01-2020\_thru\_12-31-2024\_AS\_OF\_10-10-2025.csv} $\rightarrow$ \texttt{data/raw\_data/}
\item \texttt{zip\_code\_database.csv} $\rightarrow$ \texttt{data/raw\_data/}
\end{itemize}
\end{enumerate}
\vspace{0.5cm}
Step 3: Run Data Preparation
\begin{enumerate}
\item Execute \texttt{data\_prep\_for\_jds\_datacleansing.R} in RStudio
\item The script will automatically create directory structure and install missing packages
\item \textit{Expected runtime: Approximately 90 minutes on a system with 16 GB RAM}
\item \textit{Note: Initial data loading is memory-intensive; the script optimizes memory usage after initial processing}
\item Upon completion, verify these files exist:
\begin{itemize}
\item \texttt{data/5-year\_311SR\_01-01-2020\_thru\_12-31-2024\_AS\_OF\_10-10-2025.rds}
\item \texttt{data/USPS\_zipcodes.rds}
\item \texttt{console\_output/JDS\_data\_prep\_console\_output.txt}
\end{itemize}
\end{enumerate}
\vspace{0.5cm}
Step 4: Run Data Cleansing Analysis
\begin{enumerate}
\item Execute \texttt{code/jds\_datacleansing.R} in RStudio
\item \textit{Expected runtime: Approximately 60 minutes on a system with 16 GB RAM}
\item Monitor progress in RStudio's plot pane as charts are generated
\item The script will produce:
\begin{itemize}
\item 82 PDF charts in \texttt{charts/}
\item \texttt{field\_usage\_summary\_table.csv} in \texttt{data/analytical\_files/}
\item \texttt{console\_output/JDS\_datacleaning\_console\_output.txt}
\end{itemize}
\end{enumerate}

\subsection*{Validation}
Compare your console output files against the reference outputs provided in the 
GitHub repository to verify successful reproduction. The console outputs contain 
detailed execution logs, row counts, and session information.

\subsection*{Notes}
\begin{itemize}
\item Total processing time: Approximately 2.5 hours on a system with 16 GB RAM; faster on systems with 32 GB
\item The scripts use \texttt{data.table} for memory-efficient processing of 10+ million records
\item All file paths are configured relative to \texttt{base\_dir} for portability
\item Memory management: Initial data loading requires substantial RAM; the script reduces memory footprint after preprocessing
\end{itemize}


\subsection*{Validation}
Compare your console output files against the reference outputs provided 
in the GitHub repository to verify successful reproduction. The console outputs contain
tables of information which should match your run. 

\subsection*{Notes}
\begin{itemize}
\item Total processing time: Approximately 2.5 hours on a system with 16 GB RAM; faster on systems with 32 GB
\item The scripts use \texttt{data.table} for memory-efficient processing of 10+ million records
\item All file paths are configured relative to \texttt{base\_dir} for portability
\item Memory management: Initial data loading requires substantial RAM; the script reduces memory footprint after preprocessing
\end{itemize}



%---------------------------------------------------------------
\section{Supplementary Discussion}

Provide extended interpretations, sensitivity analyses, or additional
background not appropriate for the main text but useful for readers
seeking deeper explanation.

%---------------------------------------------------------------
\section{References}

Use the same citation style as the main manuscript. For example:



\end{document}

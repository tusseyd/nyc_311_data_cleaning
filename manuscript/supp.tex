\documentclass[12pt]{article}

\usepackage{amsmath, amssymb, graphicx, booktabs, url, hyperref}
\usepackage{setspace}
\onehalfspacing
\usepackage[margin=1in]{geometry}

\usepackage{siunitx} % For alignment of numbers
\sisetup{
    group-separator = {,},
    round-mode = places,
    round-precision = 2,
    output-decimal-marker = {.},
    table-number-alignment = center,
    table-figures-integer = 6,
    table-figures-decimal = 2,
    table-figures-uncertainty = 2
}

\usepackage{comment}
\usepackage{multicol}
\usepackage{fontawesome5}
\usepackage{booktabs, textgreek}
\usepackage{subcaption}
\usepackage{enumitem}
\usepackage{tikz} % For creating diagrams
\usepackage{hyperref}   % For clickable links and breaking long URLs
\usetikzlibrary{positioning} % Required for relative positioning (e.g., "of" keyword)

% image path
\graphicspath{{./}{./supplemental_images/}}

% Number figures/tables as S1, S2, ... instead of 1, 2, ...
\renewcommand{\thefigure}{S\arabic{figure}}
\renewcommand{\thetable}{S\arabic{table}}

%% float control
\renewcommand\floatpagefraction{0.75}
% \renewcommand\topfraction{.8}
% \renewcommand\bottomfraction{.8}
% \renewcommand\textfraction{.2}
\setcounter{totalnumber}{50}
\setcounter{topnumber}{50}
\setcounter{bottomnumber}{50}

\hypersetup{
  colorlinks=true,
  linkcolor=blue,
  citecolor=blue,
  urlcolor=blue
}

\hyphenpenalty=950

\begin{document}
\doublespacing 

\maketitle
% --- Front matter lists ---
\tableofcontents
\clearpage    % start lists on a new page (optional)

\listoffigures
\clearpage

\listoftables
\clearpage

\begin{frontmatter}
\title{Supplementary Materials for:\\
\textbf{Principles for Open Data Curation: A Case Study with the New York City 311 Service Request Data}}


\author[1]{\inits{D.}\fnms{David}~\snm{Tussey}}
\author[2]{\inits{J.}\fnms{Jun}~\snm{Yan}}
\address[1]{\institution{NYC Department of Information Technology and Telecommunications}, \cny{USA}}
\address[2]{Department of Statistics,
  \institution{University of Connecticut}, \cny{USA}}

%\date{\today}
\begin{keywords}
  \kwd{Data cleansing}
  \kwd{Data quality}
  \kwd{Data science}
  \kwd{Data Validation}
  \kwd{NYC Open Data}
\end{keywords}

\end{frontmatter}


%---------------------------------------------------------------
\section{Overview}
This document provides supplementary materials accompanying the article
\emph{<Principles for Open Data Curation: A Case Study with the New York City 311 Service Request Data>}. 
It includes additional figures, tables,extended methodological details, 
data-processing steps, console output, and reproducibility
information that support but are not included in the main manuscript.


%---------------------------------------------------------------
\section{Additional Figures}
\subsection{Figure S1: <Description>}
\begin{figure}[h!]
  \centering
  \includegraphics[width=0.85\textwidth]{images/figure_S1.pdf}
  \caption{<Full caption for the figure.>}
\end{figure}

\subsection{Figure S2: <Description>}
<Repeat as needed>

%---------------------------------------------------------------
\section{Additional Tables}
\subsection{Table S1: <Description>}
\begin{table}[h!]
\centering
\begin{tabular}{lrrr}
\toprule
Variable & Mean & SD & N \\
\midrule
<...> \\
\bottomrule
\end{tabular}
\caption{<Full caption for the table.>}
\end{table}

%---------------------------------------------------------------
\section{Extended Methodological Details}
\subsection{Control Chart Computation}
Provide details here that were omitted from the main text due to space.
For example, formula derivations, parameter choices, model assumptions,
or additional diagnostic checks.

\subsection{Data Quality Metrics}
Explain definitions of metrics, edge-case handling, and any technical
decisions (e.g., treatment of DST, Feb.~29 adjustments, outlier winsorization).

%---------------------------------------------------------------
\section{Reproducibility and Code Access}
All analysis code is available at:
\begin{itemize}
  \item \textsc{R} source code: \url{https://github.com/tusseyd/nyc_311_data_cleaning}
  \item USPS Zip Code reference: \url{https://www.unitedstateszipcodes.org/zip-code-database/}
  \item NYC 311 dataset (Figshare): \url{<your final figshare link>}
\end{itemize}

Optionally include code excerpts:

\subsection{Key \textsc{R} Functions}
\begin{verbatim}
# Example header for a function
plot_duration_histogram <- function(DT, value_col, ...) {
    ...
}
\end{verbatim}

%---------------------------------------------------------------
\section{Supplementary Discussion}
Provide extended interpretations, sensitivity analyses, or additional
background not appropriate for the main text but useful for readers
seeking deeper explanation.

%---------------------------------------------------------------
\section{References}
Use the same citation style as the main manuscript. For example:

\begin{itemize}
%\item Grant, E.\ L., and Leavenworth, R.\ S.\ (1996). \textit{Statistical Quality Control}. 7th Edition.
%\item Montgomery, D.\ C.\ (2020). \textit{Introduction to Statistical Quality Control}. Wiley.
\end{itemize}

\end{document}
